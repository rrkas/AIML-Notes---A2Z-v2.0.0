\chapter{Introduction to Information Retrieval}

\section{What is Information Retrieval? \cite{gfg-what-is-ir}}
Information Retrieval (IR) can be defined as a software program that deals with the organization, storage, retrieval, and evaluation of information from document repositories, particularly textual information. Information Retrieval is the activity of obtaining material that can usually be documented on an unstructured nature i.e. usually text which satisfies an information need from within large collections which is stored on computers. For example, Information Retrieval can be when a user enters a query into the system. 

\section{Information Retrieval VS Data Retrieval \cite{gfg-what-is-ir}}
\begin{longtable}[H]{|p{7.5cm}|p{7.5cm}|}
    \caption{Information Retrieval VS Data Retrieval}\\
    \hline

    \textbf{Information Retrieval} & \textbf{Data Retrieval}  \\
    \hline
    \endfirsthead

    \textbf{Information Retrieval} & \textbf{Data Retrieval}  \\
    \hline\endhead
    \hline\endfoot
    \hline\endlastfoot
     
     \hline
     The software program that deals with the organization, storage, retrieval, and evaluation of information from document repositories particularly textual information.  & Data retrieval deals with obtaining data from a database management system such as ODBMS. It is A process of identifying and retrieving the data from the database, based on the query provided by user or application. \\
     \hline
     Retrieves information about a subject. & Determines the keywords in the user query and retrieves the data. \\
     \hline
     Small errors are likely to go unnoticed. & A single error object means total failure. \\
     \hline
     Not always well structured and is semantically ambiguous. & Has a well-defined structure and semantics. \\
     \hline
     Does not provide a solution to the user of the database system. & Provides solutions to the user of the database system. \\
     \hline
     The results obtained are approximate matches. & The results obtained are exact matches. \\
     \hline
     Results are ordered by relevance. & Results are not ordered by relevance.\\
     \hline
     It is a probabilistic model. & It is a deterministic model.\\
     \hline
\end{longtable}


\section{Taxonomy of Information Retrieval \cite{researchgate/47397195_A_Taxonomy_of_Information_Retrieval_Models_and_Tools}}

\subsection{vertical taxonomy}

A vertical taxonomy classifies IR models with respect to a set of basic features.
The vertical taxonomy is built by exploding two basic features of any IR model: the \textbf{representation}, that is the model adopted to represent both the documents and the user queries; and the \textbf{reasoning}, which refers to the framework adopted to resolve a representation similarity problem. 

\subsection{horizontal taxonomy}

A horizontal taxonomy classifies IR objects with respect to their tasks, form, and context. 
The horizontal taxonomy is derived from an analysis of the application areas of IR.










































