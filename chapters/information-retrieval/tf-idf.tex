\chapter{Term Frequency-Inverse Document Frequency (TF-IDF) \cite{nlp-1}} \label{Term Frequancy-Inverse Document Frequency (TF-IDF)}

\url{https://drive.google.com/file/d/14x7oawk84MPvBV7MEU7Q6ovVudMfUHBA/view}\\
121/577

\section{Vectors and documents \cite{nlp-1}}\label{tf-idf: Vectors and documents}

\begin{itemize}
    \item In a \textbf{term-document matrix}\index{term-document matrix} \label{term-document matrix}, each row represents a word in the vocabulary and each column represents a document from some collection of documents.
    
    \item The term-document matrix was first defined as part of the \textbf{vector space model}\index{vector space model}\label{vector space model} of information retrieval. In this model, a document is represented as a count vector.

    \item in term-document matrices, the vectors representing each document would have dimensionality $|V|$ (vocabulary size).

    \item Two documents that are similar will tend to have similar words, and if two documents have similar words their column vectors will tend to be similar.

    \item documents can be represented as vectors in a vector space. A \textbf{vector space} (SEE: \nameref{Vector Spaces}) is a collection of vectors, characterized by their dimension.
\end{itemize}

\section{Words as vectors: document dimensions \cite{nlp-1}}\label{tf-idf: Words as vectors: document dimensions}

\begin{itemize}
    \item Vector semantics can also be used to represent the meaning of words. 

    \item We do this row vector by associating each word with a word vector - a \textbf{row vector} rather than a column vector, hence with different dimensions.

    \item similar words have similar vectors because they tend to occur in similar documents. The term-document matrix thus lets us represent the meaning of a word by the documents it tends to occur in.
\end{itemize}


\section{Words as vectors: word dimensions \cite{nlp-1}} \label{tf-idf: Words as vectors: word dimensions}

\begin{itemize}
    \item \textbf{term-term matrix/ word-word matrix/ term-context matrix}\index{term-term matrix/ word-word matrix/ term-context matrix}\label{term-term matrix/ word-word matrix/ term-context matrix}, in which the columns are labeled by words rather than documents. 
    
    \item Dimensionality $|V|\times|V|$ and each cell records the number of times the row (target) word and the column (context) word co-occur in some context in some training corpus.

    \item The context could be the document, in which case the cell represents the number of times the two words appear in the same document.

    \item It is most common, however, to use smaller contexts, generally a window around the word, for example of $4$ words to the left and $4$ words to the right, in which case the cell represents the number of times (in some training corpus) the column word occurs in such a $\pm 4$ word window around the row word.

    \item Since most of the entries of the matrix are zero these are \textbf{sparse vector representations}.
\end{itemize}


\section{Cosine Similarity \cite{nlp-1}}\label{Cosine Similarity}























































































































































