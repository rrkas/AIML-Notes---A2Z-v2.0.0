\chapter{Artificial Intelligence (AI) \cite{aci-1}}

\begin{enumerate}[itemsep=0.2cm]
    \item The field of artificial intelligence, or AI, attempts not just to understand but also to build intelligent entities. 

    \item \textbf{Acting humanly}: \textit{Turing test}
    \begin{enumerate}
        \item The \textbf{Turing Test}\indexlabel{Turing Test}, proposed by Alan Turing (1950), was designed to provide a satisfactory operational definition of intelligence.\\
        A computer passes the test if a human interrogator, after posing some written questions, cannot tell whether the written responses come from a person or from a computer.\\
        total Turing Test includes a video signal so that the interrogator can test the subject’s perceptual abilities, as well as the opportunity for the interrogator to pass physical objects “through the hatch.”\\
        To pass the total Turing Test, the computer will need:
        \begin{enumerate}
            \item \textbf{computer vision} to perceive objects

            \item \textbf{robotics} to manipulate objects and move about
        \end{enumerate}

        \item The computer would need to possess the following capabilities: 
        \begin{enumerate}
            \item \textbf{natural language processing} to enable it to communicate successfully in English

            \item \textbf{knowledge representation} to store what it knows or hears

            \item \textbf{automated reasoning} to use the stored information to answer questions and to draw new conclusions

            \item \textbf{machine learning} to adapt to new circumstances and to detect and extrapolate patterns
        \end{enumerate}
    \end{enumerate}

    \item \textbf{Thinking humanly}: \textit{The cognitive modeling approach} 
    \begin{enumerate}
        \item The interdisciplinary field of \textbf{cognitive science}\indexlabel{cognitive science} brings together computer models from AI and experimental techniques from psychology to construct precise and testable theories of the human mind. 

        \item an algorithm performs well on a task and that it is therefore a good model of human performance, or vice versa.
    \end{enumerate}

    \item \textbf{Thinking rationally}: \textit{The “laws of thought” approach}
    \begin{enumerate}
        \item The Greek philosopher Aristotle was one of the first to attempt to codify “right thinking,” that is, irrefutable reasoning processes

        \item His syllogisms/ arguments provided patterns for argument structures that always yielded correct conclusions when given correct premises—for example, \textit{“Socrates is a man; all men are mortal; therefore, Socrates is mortal.”}

        \item logicist tradition within artificial intelligence hopes to build on such programs to create intelligent systems. \\
        There are two main obstacles to this approach.
        \begin{enumerate}
            \item it is not easy to take informal knowledge and state it in the formal terms required by logical notation, particularly when the knowledge is less than 100\% certain

            \item there is a big difference between solving a problem “in principle” and solving it in practice.
        \end{enumerate}
    \end{enumerate}

    \item \textbf{Acting rationally}: \textit{The rational agent approach}
    \begin{enumerate}
        \item An \textbf{agent}\indexlabel{agent} is just something that acts (agent comes from the Latin agere, to do).\\
        Of course, all computer programs do something, but computer agents are expected to do more: operate autonomously, perceive their environment, persist over a prolonged time period, adapt to change, and create and pursue goals. 
        
        \item A \textbf{rational agent}\indexlabel{rational agent} is one that acts so as to achieve the best outcome or, when there is uncertainty, the best expected outcome.

        \item In the \textit{“laws of thought” approach} to AI, the emphasis was on correct inferences.\\
        Making correct inferences is sometimes part of being a rational agent, because one way to act rationally is to reason logically to the conclusion that a given action will achieve one’s goals and then to act on that conclusion.\\
        On the other hand, correct inference is not all of ration-ality; in some situations, there is no provably correct thing to do, but something must still be done.\\
        There are also ways of acting rationally that cannot be said to involve inference.\\
        \textbf{For example}, recoiling from a hot stove is a reflex action that is usually more successful than a slower action taken after careful deliberation.

        \item The rational-agent approach has two advantages over the other approaches. 
        \begin{enumerate}
            \item it is more general than the “laws of thought” approach because correct inference is just one of several possible mechanisms for achieving rationality. 

            \item it is more amenable to scientific development than are approaches based on human behavior or human thought. 
        \end{enumerate}

        \item The standard of rationality is mathematically well defined and completely general, and can be “unpacked” to generate agent designs that provably achieve it.\\
        Human behavior, on the other hand, is well adapted for one specific environment and is defined by, well, the sum total of all the things that humans do.

        \item \textbf{limited rationality}\indexlabel{limited rationality} - acting appropriately when there is not enough time to do all the computations one might like.
    \end{enumerate}
\end{enumerate}

\section{Fundamentals of AI \cite{aci-1}}

\begin{enumerate}[itemsep=0.2cm]
    \item \textbf{Philosophy}
    % \begin{enumerate}
    %     \item  Can formal rules be used to draw valid conclusions?
    %     \begin{enumerate}
            % \item \textbf{Aristotle} (384–322 B.C.) was the first to formulate a precise set of laws governing the rational part of the mind. \\
            % He developed an informal system of syllogisms for proper reasoning, which in principle allowed one to generate conclusions mechanically, given initial premises. 

            % \item \textbf{Ramon Lull} (d. 1315) had the idea that useful reasoning could actually be carried out by a mechanical artifact.

            % \item \textbf{Thomas Hobbes} (1588–1679) proposed that reasoning was like numerical computation, that “we add and subtract in our silent thoughts.”

            % \item \textbf{Leonardo da Vinci} (1452–1519) designed but did not build a mechanical calculator; recent reconstructions have shown the design to be functional. 

            % \item The first known calculating machine was constructed around 1623 by the German scientist \textbf{Wilhelm Schickard} (1592–1635).
            
            % \item \textit{Pascaline}, built in 1642 by \textbf{Blaise Pascal} (1623–1662), is more famous.

            % \item \textbf{Gottfried Wilhelm Leibniz} (1646–1716) built a mechanical device intended to carry out operations on concepts rather than numbers, but its scope was rather limited.

        %     \item 
        % \end{enumerate}

    %     \item How does the mind arise from a physical brain? 

    %     \item Where does knowledge come from? 

    %     \item How does knowledge lead to action?
    % \end{enumerate}

    \item \textbf{Mathematics}
    % \begin{enumerate}
    %     \item What are the formal rules to draw valid conclusions?

    %     \item What can be computed?

    %     \item How do we reason with uncertain information?
    % \end{enumerate}

    \item \textbf{Economics}
    % \begin{enumerate}
    %     \item How should we make decisions so as to maximize payoff?

    %     \item How should we do this when others may not go along?

    %     \item How should we do this when the payoff may be far in the future?
    % \end{enumerate}

    \item \textbf{Neuroscience}
    % \begin{enumerate}
    %     \item How do brains process information? 
    % \end{enumerate}

    \item \textbf{Psychology}
    % \begin{enumerate}
    %     \item How do humans and animals think and act?
    % \end{enumerate}

    \item \textbf{Computer engineering}
    % \begin{enumerate}
    %     \item How can we build an efficient computer? 
    % \end{enumerate}

    \item \textbf{Control theory and cybernetics}
    % \begin{enumerate}
    %     \item How can artifacts operate under their own control?
    % \end{enumerate}

    \item \textbf{Linguistics}
    % \begin{enumerate}
    %     \item How does language relate to thought?
    % \end{enumerate}

\end{enumerate}









