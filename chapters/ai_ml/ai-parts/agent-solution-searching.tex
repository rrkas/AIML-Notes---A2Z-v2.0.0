\chapter{AI Agent Solution Searching \cite{aci-1}}

\begin{enumerate}
    \item A \textbf{solution} is an action sequence, so search algorithms work by considering various possible action sequences. 
    
    \item The possible action sequences starting at the initial state form a \textbf{search tree} with:
    \begin{enumerate}
        \item the initial state at the root
        \item the branches are actions
        \item the nodes correspond to states in the state space of the problem
    \end{enumerate}

    \item \textbf{expanding} the current state: applying each legal action to the current state, thereby \textbf{generating} a new set of states

    \item  essence of search—following up one option now and putting the others aside for later, in case the first choice does not lead to a solution.

    \item \textbf{leaf node}: a node with no children in the tree.

    \item \textbf{frontier}/ \textbf{open list}: The set of all leaf nodes available for expansion at any given point\\
    the frontier separates the state-space graph into the explored region and the unexplored region, so that every path from the initial state to an unexplored state has to pass through a state in the frontier

    \item \textbf{explored set} / \textbf{closed list}: remembers every expanded node

    \item The process of expanding nodes on the frontier continues until either a solution is found or there are no more states to expand.

    \item \textbf{search strategy}: how they choose which state to expand next
\end{enumerate}


SEE:
\begin{enumerate}
    \item \fullref{Tree Search (generic)}
    \item \fullref{Graph Search (generic)}
    
    \item Uninformed:
    \begin{enumerate}
        \item \fullref{Breadth-first search BFS}
        \item \fullref{Uniform-cost search (UCS)}
        \item \fullref{Depth-first search (DFS)}
        \item \fullref{Depth-limited search (DLS)}
        \item \fullref{Iterative deepening depth-first search/ Iterative deepening search (IDS)}
        \item \fullref{Iterative Lengthening Search (ILS)}
        \item \fullref{Bidirectional search}
    \end{enumerate}
    
    \item Informed:
    \begin{enumerate}
        \item \fullref{Best-first Search}
        \item \fullref{Greedy best-first search (GBFS)}
        \item \fullref{A* (A-star) search}
        \item \fullref{Iterative-Deepening A* (IDA*) Search}
        \item \fullref{Recursive best-first search (RBFS)}
        \item \fullref{Memory-bounded A* (MA*)}
        \item \fullref{Simplified Memory-bounded A* (SMA*)}
    \end{enumerate}
\end{enumerate}


\section{Node of a tree/ graph (n)}
\begin{table}[H]
    \centering
    \begin{tabular}{l p{8cm}}
        \textit{n}.STATE & the state in the state space to which the node corresponds \\
        
        \textit{n}.PARENT & the node in the search tree that generated this node\\
        
        \textit{n}.ACTION &  the action that was applied to the parent to generate the node \\
        
        \textit{n}.PATH-COST & the cost, traditionally denoted by $g(n)$, of the path from the initial state to the node, as indicated by the parent pointers.
    \end{tabular}
\end{table}


\begin{algorithm}
    \caption{ function CHILD-NODE takes a parent node and an action and returns the resulting child node}

    \SetKwFunction{FUNCTION}{CHILD-NODE}
    \SetKwProg{Fn}{function}{:}{}
    \Fn{\FUNCTION{problem, parent, action}}{
        \Return a node with \\
        \hspace{0.4cm} STATE = \textit{problem}.RESULT(\textit{parent}.STATE, \textit{action}), \\
        \hspace{0.4cm} PARENT = \textit{parent},  \\
        \hspace{0.4cm} ACTION = \textit{action},\\
        \hspace{0.4cm} PATH-COST = \textit{parent}.PATH-COST + \textit{problem}.STEP-COST(\textit{parent}.STATE, \textit{action})
    }
\end{algorithm}





































