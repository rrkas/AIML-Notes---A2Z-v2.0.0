\chapter{Vector \cite{mfml-1}}\label{chapter: Vector}

\section*{Intro \cite{mfml-1}}
\begin{table}[H]
    \begin{tabular}{l l}
        rows/ row vectors & $(1, n)$-matrices are called rows \\

        columns/ column vectors & $(m, 1)$-matrices are called columns \\
    \end{tabular}
\end{table}

Alternate Definition: SEE \fullref{vectors}


\section{Outer Product ( $a\otimes b= \mathbf{ab^\top} \in \mathbb{R}^{m\times n}$ ) \cite{mfml-1}} \label{vector: Outer Product}
\[
    a\otimes b= \mathbf{ab^\top} = 
    \begin{bmatrix}
        a_1\\
        \vdots\\
        a_m
    \end{bmatrix}
    \begin{bmatrix}
        b_n & \cdots & b_n
    \end{bmatrix}
    =
    \begin{bmatrix}
        a_1 b_1 & a_1 b_2 & \cdots & a_1 b_n\\
        a_2 b_1 & a_2 b_2 & \cdots & a_2 b_n\\
        \vdots & \vdots & \ddots & \vdots \\
        a_m b_1 & a_m b_2 & \cdots & a_m b_n\\
    \end{bmatrix} \in \mathbb{R}^{m\times n}
\]
\[
    (a\otimes b)_{ij} = a_i b_j
\]

\section{Inner/ Scalar /Dot Product ( $\langle a, b \rangle =a\cdot b = \mathbf{a^\top b} \in \mathbb{R}$ ) \cite{mfml-1,wiki/Dot_product}} \label{vector: Inner/ Scalar /Dot Product}

The dot product or scalar product is an algebraic operation that takes two vectors of equal length, and returns a \textbf{single number}.
the dot product between two vectors $a, b$ is denoted by $a^\top b$ or $\langle a, b\rangle \in R$:
\[
    \displaystyle
    \langle a, b \rangle =
    a\cdot b = 
    \mathbf{a^\top b} =
    \sum_{i=1}^{n} a_i b_i
    \in \mathbb{R}
\]


\section{Codirection}\label{Codirection}
Two vectors that point in the \textbf{same direction} are called codirected. 


\section{Collinearity}\label{Collinearity}
Two vectors are collinear if they point in the \textbf{same or the opposite direction}.


\section{Partial Differentiation ( $\partial f/\partial x$ )} \label{vectors: Partial Differentiation}

For a function $f : R^n \to R$, $x \mapsto f(x)$, $x \in R^n$ of $n$ variables $x_1, \cdots , x_n$ we define the partial derivatives as
\[
    f = f(x) = f(x_1, \cdots , x_n)
\]
\[
    \displaystyle
    \begin{matrix}
    \dfrac{\partial f}{\partial x_1} = 
    \lim_{h\to 0} \dfrac{f(x_1+h, \cdots , x_n) - f(x)}{h} \\
    \vdots\\
    \dfrac{\partial f}{\partial x_n} = 
    \lim_{h\to 0} \dfrac{f(x_1, \cdots , x_n+h) - f(x)}{h} \\
    \end{matrix}
\]

where $n$ is the number of variables and $1$ is the dimension of the \textbf{image/ range/ codomain} of $f$ \indexlabel{function of vector: image/ range/ codomain}.


\subsection{Rules of Partial Differentiation}

\begin{table}[h]
    \centering
    \begin{tabular}{|p{2.5cm}|p{12.5cm}|}
        \hline
        Sum Rule & \begin{minipage}{12cm}
            \vspace{-0.1cm}
            \[
                \dfrac{\partial}{\partial x}(f(x) + g(x))
                = 
                \dfrac{\partial f(x)}{\partial x} +
                \dfrac{\partial g(x)}{\partial x}
            \]
            \vspace{0.1cm}
        \end{minipage} \\
        \hline
        Product Rule & \begin{minipage}{12cm}
            \vspace{-0.1cm}
            \[
                \dfrac{\partial}{\partial x}(f(x)g(x))
                = 
                \dfrac{\partial f(x)}{\partial x}g(x) +
                f(x)\dfrac{\partial g(x)}{\partial x}
            \]
            \vspace{0.1cm}
        \end{minipage} \\
        \hline
        Chain Rule & \begin{minipage}{12cm}
            \vspace{-0.1cm}
            \[
                \dfrac{\partial}{\partial x}((g \circ f)(x))
                = 
                \dfrac{\partial}{\partial x}(g(f(x))
                =
                \dfrac{\partial g(x)}{\partial f(x)}
                \dfrac{\partial f(x)}{\partial x}
            \]
            
            Consider a function $f(x_1, x_2) = f : R^2 \to R$ of two variables $x_1, x_2$. Furthermore, $x_1(t)$ and $x_2(t)$ are themselves functions of $t$. To compute the gradient of $f$ with respect to $t$:
            \[
                \dfrac{df}{dt}
                = \begin{bmatrix}
                    \dfrac{\partial f}{\partial x_1} &
                    \dfrac{\partial f}{\partial x_2}
                \end{bmatrix}
                \begin{bmatrix}
                    \dfrac{\partial x_1(t)}{\partial t}\\
                    \dfrac{\partial x_2(t)}{\partial t}
                \end{bmatrix}
                = \dfrac{\partial f}{\partial x_1}
                \dfrac{\partial x_1}{\partial t}
                +
                \dfrac{\partial f}{\partial x_2}
                \dfrac{\partial x_2}{\partial t}
            \]

            If $f(x_1, x_2)$ is a function of $x_1$ and $x_2$, where $x_1(s, t)$ and $x_2(s, t)$ are themselves functions of two variables $s$ and $t$, the chain rule yields the partial derivative:
            \[
                \dfrac{df}{ds}
                = \dfrac{\partial f}{\partial x_1}
                \dfrac{\partial x_1}{\partial s}
                +
                \dfrac{\partial f}{\partial x_2}
                \dfrac{\partial x_2}{\partial s}
            \]
            \[
                \dfrac{df}{dt}
                = \dfrac{\partial f}{\partial x_1}
                \dfrac{\partial x_1}{\partial t}
                +
                \dfrac{\partial f}{\partial x_2}
                \dfrac{\partial x_2}{\partial t}
            \]
            \vspace{0.1cm}
        \end{minipage} \\
        \hline
    \end{tabular}
\end{table}


\section{Gradient of vector function ( $\nabla _xf = \operatorname{grad} f$ )}\label{Gradient of vector function}

For a function $f : R^n \to R^n$, $x \to f(x)$, $x \in R^n$ of $n$ variables $x_1, \cdots , x_n$ we define the gradient as

\[
    \nabla_x f = 
    \operatorname{grad} f = 
    \dfrac{df}{dx} =
    \begin{bmatrix}
        \dfrac{\partial f(x)}{\partial x_1} &
        \dfrac{\partial f(x)}{\partial x_2} &
        \cdots &
        \dfrac{\partial f(x)}{\partial x_n}
    \end{bmatrix}
    \in R^{1\times n}
\]

\noindent where,
\begin{enumerate}
    \item $n$ is the number of variables and $1$ is the dimension of the \textbf{image/ range/ codomain} of f \indexlabel{gradient of vector fn: image/ range/ codomain}. 

    \item It is collection of partial derivatives  

    \item The row vector is called the gradient of $f$ or the \textbf{Jacobian} and is the generalization of the derivative.

    \item This definition of the Jacobian is a special case of the general definition of the Jacobian for vector-valued functions as the collection of partial
\end{enumerate}

If $f(x_1, x_2)$ is a function of $x_1$ and $x_2$, where $x_1(s, t)$ and $x_2(s, t)$ are themselves functions of two variables $s$ and $t$, the chain rule yields the gradient:
\[
    \renewcommand{\arraystretch}{2}
    \dfrac{df}{d(s,t)} =
    \dfrac{\partial f}{\partial x}
    \dfrac{\partial x}{\partial (s,t)} =
    \begin{bmatrix}
        \dfrac{\partial f}{\textcolor{blue}{\partial x_1}} &
        \dfrac{\partial f}{\textcolor{Maroon}{\partial x_2}}
    \end{bmatrix}
    \begin{bmatrix}
        \textcolor{blue}{\dfrac{\partial x_1}{\partial s}} &
        \textcolor{blue}{\dfrac{\partial x_1}{\partial t}}\\
        \textcolor{Maroon}{\dfrac{\partial x_2}{\partial s}} &
        \textcolor{Maroon}{\dfrac{\partial x_2}{\partial t}}\\
    \end{bmatrix}
\]

\section{Vector-Valued Functions}\label{Vector-Valued Functions}
For a function $f : Rn -> Rm$ and a vector $x = [x_1, \cdots , x_n]^\top \in R^n$, the corresponding vector of function values is given as:
\[
    f(x) = 
    \begin{bmatrix}
        f_1(x)\\
        \vdots\\
        f_m(x)\\
    \end{bmatrix}
    \in R^m
\]

where:
\begin{enumerate}
    \item $f =  [f_1, \cdots , f_m]^\top$

    \item $f_i$ or $f_i(x) : R^n \to R$ that maps $x$ onto $R$
\end{enumerate}

\section{Partial Derivative of Vector-Valued Functions ( $\partial f(x)/\partial x_i$ )}\label{Partial Derivative of Vector-Valued Functions}

\[
    \dfrac{\partial f}{\partial x_i}=
    \begin{bmatrix}
        \dfrac{\partial f_1}{\partial x_i} \\
        \vdots \\
        \dfrac{\partial f_m}{\partial x_i}
    \end{bmatrix} =
    \begin{bmatrix}
        \displaystyle\lim_{h \to 0} \dfrac{ f_1(x_1,\cdots ,x_{i-1},x_i+h,x_{i+1},\cdots x_n) - f_1(x)}{h} \\
        \vdots \\
        \displaystyle\lim_{h \to 0} \dfrac{f_m(x_1,\cdots ,x_{i-1},x_i+h,x_{i+1},\cdots x_n) - f_m(x)}{h}
    \end{bmatrix}
    \in R^m
\]


\section{Jacobian/ Gradient of Vector-Valued Functions ( $J = \nabla_x f = df(x)/dx$ )}\label{Jacobian/ Gradient of Vector-Valued Functions}

The collection of all \textbf{first-order partial derivatives} of a vector-valued function $f : R^n \to R^m$ is called the Jacobian. The Jacobian $J$ is an $m \times n$ matrix, which we define and arrange as follows:

\[
    J = \nabla_x f =
    \dfrac{df(x)}{dx} =
    \begin{bmatrix}
        \textcolor{blue}{\dfrac{\partial f(x)}{\partial x_1}} &
        \cdots &
        \textcolor{Maroon}{\dfrac{\partial f(x)}{\partial x_n}}
    \end{bmatrix} =
    \begin{bmatrix}
        \textcolor{blue}{\dfrac{\partial f_1(x)}{\partial x_1}} &
        \cdots &
        \textcolor{Maroon}{\dfrac{\partial f_1(x)}{\partial x_n}}\\
        \vdots & \ddots & \vdots \\
        \textcolor{blue}{\dfrac{\partial f_m(x)}{\partial x_1}} &
        \cdots &
        \textcolor{Maroon}{\dfrac{\partial f_m(x)}{\partial x_n}}\\
    \end{bmatrix}
    \in R^{m\times n}
\]

\begin{enumerate}
    \item we use the numerator layout of the derivative, i.e., the derivative $df/dx$ of $f \in R^m$ with respect to $x \in R^n$ is an $m \times n$ matrix, where the elements of $f$ define the rows and the elements of $x$ define the columns of the corresponding Jacobian

    \item there also exists also the denominator layout, which is the transpose of the numerator layout

    \item default = numerator layout

    \item Jacobian determinant $|det(J)|$ is the factor by which areas or volumes are scaled when coordinates are transformed.
\end{enumerate}


\section{Higher-Order Derivatives}\label{Higher-Order Derivatives}

\begin{table}[H]
    \centering
    \renewcommand{\arraystretch}{2.5}
    \begin{tabular}{|p{3cm}|p{12cm}|}
        \hline
        $\dfrac{\partial^2 f}{\partial x^2}$ & second partial derivative of $f$ with respect to $x$ \\
        \hline
        $\dfrac{\partial^n f}{\partial x^n}$ & $n$th partial derivative of $f$ with respect to $x$ \\
        \hline
        $\dfrac{\partial^2 f}{\partial y \partial x} = \dfrac{\partial}{\partial y}\left( \dfrac{\partial f}{\partial x} \right)$ & partial derivative obtained by first partial differentiating with respect to $x$ and then with respect to $y$ \\
        \hline
        $\dfrac{\partial^2 f}{\partial x \partial y} = \dfrac{\partial}{\partial x}\left( \dfrac{\partial f}{\partial y} \right)$ & partial derivative obtained by first partial differentiating by $y$ and then $x$ \\
        \hline
    \end{tabular}
    \caption{Higher-Order Derivatives}
\end{table}

\begin{enumerate}
    \item If $f(x, y)$ is a twice (continuously) differentiable function, then 
    $\dfrac{\partial^2 f}{\partial x \partial y} = \dfrac{\partial^2 f}{\partial y \partial x}$

\end{enumerate}


\section{Hessian/ Hessian Matrix ( $H = \nabla_{x,y}^2f$ )}\label{Hessian/ Hessian Matrix}
The Hessian is the collection of all \textbf{second-order partial derivative}.
\[
    H = 
    \begin{bmatrix}
        \dfrac{\partial^2 f}{\partial x^2} &
        \dfrac{\partial^2 f}{\partial x \partial y}\\
        \dfrac{\partial^2 f}{\partial y \partial x} &
        \dfrac{\partial^2 f}{\partial y^2}
    \end{bmatrix}
\]


\begin{enumerate}
    \item H is symmetric

    \item for $x \in R^n$ and $f : R^n \to R$, the Hessian is an $n \times n$ matrix

    \item The Hessian measures the curvature of the function locally around $(x, y)$.

    \item (\textbf{Hessian of a Vector Field}\indexlabel{Hessian of a Vector Field}). If $f : R^n \to R^m$ is a \textbf{vector field}\indexlabel{vector field}, the Hessian is an $(m \times n \times n)$-tensor
\end{enumerate}



\section{Gradient/ Derivative Common Formulas}\label{matrix-vector: Gradient/ Derivative Common Formulas}

\begin{enumerate}
    \subsection{Matrix-vector}
    
    \item \( d(Ax)/dx = A \)

    \item $d(Ax)/dA = v^\top\otimes I $
    
    \item \( \partial (A-1y)/\partial y = A-1 \)

    \subsection{matrix-matrix}
    \item \(
        d(XX)/dX = A^\top \otimes I+I\otimes A
    \)
    
    \item  $d(X^\top X)/dX = ??$
    
    \item  $d(XXT)/dX = ??$

    \subsection{vector-vector}
    \item  $\partial (a^\top x)/\partial x = a^\top$

    \item $\partial (x^\top a)/\partial x = a^\top$

    \item  $d(x^\top x)/dx = 2x$

    \item  $d(xx^\top)/dx = ??$

    \item  \(
        \partial (x^\top Bx)/\partial x = \begin{cases}
            x^\top(B + B^\top) & \text{ general}\\
            2xTB & \text{ $B$ is symmetric}
        \end{cases}
    \)

    \item  $\partial (a^\top X^{-1}b)/\partial X = -(X^{-1})^\top ab^\top (X^{-1})^\top$

    \item  $d(xTAx)/dA = xxT$

    \item  $\partial (a^\top Xb)/\partial X = ab^\top$

    \item  $\partial [(x - As)^\top W(x - As)]/\partial s = -2(x - As)^\top WA$  ($W$ is symmetric)

% \partial (f(X)⊤)/\partial X = (\partial f(X)/\partial X)⊤
% \partial (f(X)-1)/\partial X = -f(X)-1 ⋅ (\partial f(X)/\partial X) ⋅ f(X)-1
% \partial [tr(f(X))]/\partial X = tr(\partial f(X)/\partial X)
% \partial [det(f(X))]/\partial X = det(f(X)) ⋅ tr(f(X)-1 ⋅ (\partial f(X)/\partial X))


\end{enumerate}



















