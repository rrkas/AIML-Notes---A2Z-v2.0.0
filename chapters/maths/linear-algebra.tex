\chapter{Linear Algebra \cite{mfml-1}}

\section{Groups ( $G = (\mathbb{G}, \otimes)$ ) \cite{mfml-1}}\label{lin-alg-Groups}
Consider a set $\mathbb{G}$ and an operation $\otimes$ : $\mathbb{G} \times \mathbb{G} \to \mathbb{G}$ group defined on $\mathbb{G}$. Then $G := (\mathbb{G}, \otimes)$ is called a group if the following hold:

\begin{table}[H]
    \begin{tabular}{l p{10cm}}
        Closure of $\mathbb{G}$ under $\otimes$ & $\forall x,y \in \mathbb{G} : x \otimes y \in \mathbb{G}$ \\
        Associativity & $\forall x,y,z \in \mathbb{G}:(x \otimes y) \otimes z = x \otimes (y \otimes z) $ \\
        Neutral Element & $\exists e \in \mathbb{G} \text{ , } \forall x\in \mathbb{G}: x \otimes e = e \otimes x = x$ \\
        Inverse Element & $\forall x\in \mathbb{G} \text{ , } \exists y \in \mathbb{G}: x\otimes y = y\otimes x = e$ where $e$ is the neutral element. ( $y = x^{-1}$ and $x = y^{-1}$ ) \\
    \end{tabular}
\end{table}

\begin{itemize}
    \item Inverse of element is w.r.t. operation $\otimes$, does not necessarily mean $\displaystyle\frac{1}{x}$.
\end{itemize}

Examples:
\begin{itemize}
    \item $(\mathbb{N}_0, +)$ is \textbf{NOT} a group: Although $(\mathbb{N}_0, +)$ possesses a neutral element ($0$), the inverse elements are missing. ($\mathbb{N}_0 = \mathbb{N} \cup {0} $)

    \item $(\mathbb{Z}, \cdot)$ is \textbf{NOT} a group: Although $(\mathbb{Z}, \cdot)$ contains a neutral element ($1$), the inverse elements for any $z \in \mathbb{Z}$, $z \neq \pm 1$, are missing.

    \item $(\mathbb{R}, \cdot)$ is \textbf{NOT} a group since $0$ does not possess an inverse element.

    
\end{itemize}


\section{Abelian Group \cite{mfml-1}}\label{Abelian Group}

If additionally $\forall x,y \in \mathbb{G}:x\otimes y = y\otimes x$ (commutative), then $G = (\mathbb{G}, \otimes)$ is an Abelian Group.

Examples:
\begin{itemize}
    \item $(\mathbb{Z}, +)$ is an Abelian group.
    \item $(\mathbb{R}\textbackslash {0}, ·)$ is Abelian.
    \item $(R^n, +)$, $(Z^n, +)$, $n \in N$ are Abelian if $+$ is defined componentwise, i.e., 
    \[
        (x_1, \cdots , x_n) + (y_1, \cdots , y_n) = (x_1 + y_1, \cdots , x_n + y_n)
    \]
    Then, $(x_1, \cdots , x_n)^{-1} := (-x_1, \cdots , -x_n)$ is the inverse element and $e = (0, \cdots , 0)$ is the neutral element.
    \item $(R^{m×n}, +)$, the set of $m \times n$-matrices is Abelian (with componentwise addition)
\end{itemize}

\section{General Linear Group ( $GL(n,\mathbb{R})$ ) \cite{mfml-1}}\label{General Linear Group}
The set of regular (invertible) matrices $A \in R^{n \times n}$ is a group with respect to matrix multiplication as defined in and is called general linear group $GL(n, \mathbb{R})$. However, general linear group since matrix multiplication is not commutative, the group is \textbf{NOT} Abelian.

\begin{itemize}
    \item Closure and associativity follow directly from the definition of matrix multiplication.
    
    \item \textbf{Neutral element}: The identity matrix $\mathbb{I}_n$ is the neutral element with respect to matrix multiplication "$\cdot$" in $(R^{n\times n}, \cdot)$.

    \item \textbf{Inverse element}: If the inverse exists ($\mathbf{A}$ is regular), then $\mathbf{A}^{-1}$ is the inverse element of $\mathbf{A} \in R^{n\times n}$, and in exactly this case $(R^{n\times n}, \cdot)$ is a group, called the \textbf{general linear group}.
\end{itemize}


\section{Vector Spaces ( $V = (\mathbb{V}, +, \cdot)$ ) \cite{mfml-1}}\label{Vector Spaces}

A real-valued vector space $V = (\mathbb{V}, +, \cdot)$ is a set $\mathbb{V}$ with two operations:
\begin{align}
    + &: \mathbb{V} \times \mathbb{V} \to \mathbb{V}\\ \notag
    \cdot &: \mathbb{R} \times \mathbb{V} \to \mathbb{V} \notag
\end{align}

where,
\begin{itemize}
    \item $(\mathbb{V}, +)$ is an Abelian group
    \item Distributivity:
    \begin{enumerate}
        \item $\forall \lambda \in \mathbb{R}, \mathbf{x,y}\in \mathbb{V} : \lambda\cdot(\mathbf{x+y}) = \lambda\cdot\mathbf{x} + \lambda\cdot\mathbf{y}$
        
        \item $\forall \lambda,\psi \in \mathbb{R}, \textbf{x}\in \mathbb{V} : (\lambda + \psi) \cdot\textbf{x} = \lambda\cdot\textbf{x} + \psi\cdot\textbf{x}$
        
    \end{enumerate}

    \item Associativity (outer operation): $\forall \lambda,\psi\in\mathbb{R}, \mathbf{x}\in\mathbb{V}: \lambda\cdot(\psi\cdot\mathbf{x}) = (\lambda\psi)\cdot\mathbf{x}$

    \item Neutral element with respect to the outer operation: $\forall \mathbf{x}\in\mathbb{V}:1\cdot\mathbf{x}=\mathbf{x}$
\end{itemize}
\vspace{0.3cm}

\textbf{Notes:}
\begin{itemize}
    \item The elements $\mathbf{x}\in\mathbb{V}$ are called \textbf{vectors}\indexlabel{vectors}.

    \item The neutral element of $(\mathbb{V}, +)$ is the zero vector $0 = [0, \cdots, 0]^T$ \\
    the inner operation $+$ is called \textbf{vector addition}\indexlabel{vector addition}.

    \item The elements $\lambda\in\mathbb{R}$ are called \textbf{scalars}\indexlabel{scalars} and the outer operation $\cdot$ is a multiplication by scalars.

    \item “vector multiplication” $\mathbf{ab, a, b} \in \mathbb{R}^n$, is \textbf{NOT} defined

    \item we could define an element-wise multiplication, such that $\mathbf{c = ab}$ with $c_j = a_j \times b_j$

    \item $\mathbb{V} = \mathbb{R}^n, n \in \mathbb{N}$ is a vector space with operations defined as follows:
    \begin{itemize}
        \item \textbf{Addition}:\\ $\mathbf{x+y} = (x_1, \cdots, x_n)+(y_1, \cdots, y_n) = (x_1+y_1, \cdots, x_n+y_n) \forall \mathbf{x, y} \in \mathbb{R}^n$

        \item \textbf{Multiplication by scalars}:\\ $\lambda\mathbf{x} = \lambda(x_1, \cdots, x_n) = (\lambda x_1, \cdots, \lambda x_n) \forall \lambda \in \mathbb{R}, \mathbf{x} \in \mathbb{R}^n$
        
    \end{itemize}

    \item $\mathbb{V} = \mathbb{R}^{m\times n}, m, n \in \mathbb{N}$ is a vector space with:
    \begin{itemize}
        \item \textbf{Addition} is defined element-wise $\forall \mathbf{A, B} \in \mathbb{V}$: 
        \[
            \mathbf{A + B} = \begin{bmatrix}
                a_{11} + b_{11} & \cdots & a_{1n} + b_{1n} \\
                \vdots & & \vdots \\
                a_{m1} + b_{m1} & \cdots & a_{mn} + b_{mn} \\
            \end{bmatrix}
        \]

        \item \textbf{Multiplication by scalars} $\forall \lambda\in\mathbb{R}, \mathbf{A}\in\mathbb{R}^{m\times n}$:
        \[
            \lambda\mathbf{A} = \begin{bmatrix}
                \lambda a_{11} & \cdots & \lambda a_{1n} \\
                \vdots & & \vdots \\
                \lambda a_{m1} & \cdots & \lambda a_{mn} \\
            \end{bmatrix}
        \]
    \end{itemize}

    \item The vector spaces $\mathbb{R}^n$, $\mathbb{R}^{n\times 1}$, $\mathbb{R}^{1\times n}$ are only different in the way we write vectors:
    \begin{itemize}
        \item default: n-tuples as \textbf{column vectors}\indexlabel{column vectors} ($\mathbb{R}^n = \mathbb{R}^{n \times 1}$) : $\mathbf{x}$

        \item default: n-tuples as \textbf{row vectors}\indexlabel{row vectors} ($\mathbb{R}^{1 \times n}$) : $\mathbf{x}^\top$
    \end{itemize}
\end{itemize}


\section{Vector Subspaces/ linear subspace ( $U = (\mathbb{U}, +, \cdot)$ ) \cite{mfml-1}}\label{Vector Subspaces/ linear subspace}
Let $V = (\mathbb{V}, +, \cdot)$ be a vector space and $\mathbb{U} \subseteq \mathbb{V}$, $\mathbb{U} \neq \varnothing$. Then $U = (\mathbb{U}, +, \cdot)$ is called vector subspace of $V$ (or linear subspace) if $U$ is a vector space with the vector space operations $+$ linear subspace and $\cdot$ restricted to $\mathbb{U} \times \mathbb{U}$ and $\mathbb{R} \times \mathbb{U}$. We write $U \subseteq V$ to denote a subspace $U$ of $V$.

\begin{itemize}
    \item Vector subspace inherits Abelian group properties, the distributivity, the associativity and the neutral element from the vector space.

    \item To determine whether $U = (\mathbb{U}, +, \cdot)$ is a subspace of $V$ we still do need to show:
    \begin{itemize}
        \item $\mathbb{U} \neq \varnothing$, in particular: $\mathbf{0} \in \mathbb{U}$ (zero vector)
        \item Closure of $U$:
        \begin{enumerate}
            \item With respect to the \textbf{outer operation}: $\forall\lambda \in \mathbb{R}, \forall\mathbf{x} \in \mathbb{U} : \lambda \mathbf{x} \in \mathbb{U}$
            \item With respect to the \textbf{inner operation}: $\forall\mathbf{x, y} \in \mathbb{U} : \mathbf{x + y} \in \mathbb{U}$
        \end{enumerate}
    \end{itemize}

    \item For every vector space $V$, the trivial subspaces are $V$ itself and ${\mathbf{0}}$.
    
\end{itemize}

\section{Dimension of Vector Space ( $\dim(V)$ ) \cite{mfml-1}}\label{lin-alg: Dimension-vector-space}

\begin{itemize}
    \item The dimension of $V$ is the number of basis vectors of $V$, and we write $\dim(V)$.

    \item If $U \subseteq V$ is a subspace of $V$, then $\dim(U) \leq \dim(V)$ and $\dim(U) = \dim(V)$ if and only if $U = V$.

    \item the dimension of a vector space can be thought of as the number of independent directions in this vector space.

    \item The dimension of a vector space is \textbf{NOT} necessarily the number of elements in a vector.\\
    \textbf{Example}: $V = \rcmdXspan[\begin{bmatrix}0\\1\end{bmatrix}]$ is one-dimensional, although the basis vector possesses two elements.
    
\end{itemize}


\section{Linear Combination \cite{mfml-1}}\label{Linear Combination}
Consider a vector space $V$ and a finite number of vectors $\mathbf{x_1, \cdots , x_k} \in V$ . Then, every $\mathbf{v} \in V$ of the form
\(
    \mathbf{v} = \displaystyle\sum_{i=1}^{k} \lambda_i \mathbf{x}_i \in V
\) 
with $\lambda_1, \cdots, \lambda_k \in \mathbb{R}$ is a linear combination of the vectors $\mathbf{x_1, \cdots, x_k}$.

\begin{itemize}
    \item The $\mathbf{0}$-vector can always be written as the linear combination of $k$ vectors $\mathbf{x_1, \cdots, x_k}$ because $\displaystyle\mathbf{0} = \sum_{i=0}^{k} 0\cdot\mathbf{x_i}$ is always true.
\end{itemize}

\section{Linear (In)dependence \cite{mfml-1}}\label{Linear (In)dependence}

Let us consider a vector space $V$ with $k \in \mathbb{N}$ and $\mathbf{x_1, \cdots, x_k} \in V$. If there is a non-trivial linear combination, such that $0 = \displaystyle\sum^k_{i=1} \lambda_i\mathbf{x_i}$ with at least one $\lambda_i \neq 0$, the vectors $\mathbf{x_1, \cdots, x_k}$ are \textbf{linearly dependent}. \\If only the trivial solution exists, i.e., $\lambda_1 = \cdots = \lambda_k = 0$ the vectors $\mathbf{x_1, \cdots, x_k}$ are \textbf{linearly independent}.

\vspace{0.3cm}
\noindent\textbf{Properties}:
\begin{enumerate}
    \item $k$ vectors are either linearly dependent or linearly independent. There is no third option.

    \item If at least one of the vectors $\mathbf{x_1, \cdots, x_k}$ is $\mathbf{0}$ then they are linearly dependent. The same holds if two vectors are identical.

    \item The vectors $\{\mathbf{x_1, \cdots, x_k} : \mathbf{x_i} \neq \mathbf{0}, i = 1, \cdots, k\}, k \geq 2$, are linearly dependent if and only if (at least) one of them is a linear combination of the others. In particular, if one vector is a multiple of another vector, i.e., $\mathbf{x_i} = \lambda\mathbf{x_j} , \lambda \in \mathbb{R}$ then the set $\{\mathbf{x_1, \cdots, x_k} : \mathbf{x_i} \neq \mathbf{0}, i = 1, \cdots, k\}$ is linearly dependent.

    \item A practical way of checking whether vectors $\mathbf{x_1, \cdots, x_k} \in V$ are linearly independent is to use Gaussian elimination: 
    \begin{itemize}
        \item Write all vectors as columns of a matrix A and perform Gaussian elimination until the matrix is in row echelon form (the reduced row-echelon form is unnecessary here)
        
        \item The pivot columns indicate the vectors, which are linearly independent of the vectors on the left. Note that there is an ordering of vectors when the matrix is built.

        \item The non-pivot columns can be expressed as linear combinations of the pivot columns on their left.

        \item All column vectors are linearly independent if and only if all columns are pivot columns. If there is at least one non-pivot column, the columns (and, therefore, the corresponding vectors) are linearly dependent.
    \end{itemize}

    \item Consider a vector space $V$ with $k$ linearly independent vectors $\mathbf{b_1, \cdots, b_k}$ and $m$ linear combinations
    \begin{itemize}
        \item \( \displaystyle \mathbf{x}_1 \notag = \sum_{i=1}^{k} \bm{\lambda}_{1i} \mathbf{b}_i  \quad\quad\cdots\quad\quad \mathbf{x}_m = \sum_{i=1}^{k} \bm{\lambda}_{im} \mathbf{b}_i \)

        \item $\mathbf{B} = [\mathbf{b_1, \cdots, b_k}]$ as the matrix whose columns are the linearly independent vectors $\mathbf{b_1, \cdots, b_k}$
        \[
            \mathbf{x}_j = \mathbf{B{\bm{\lambda}}}_j \quad\quad\quad\quad \bm{\lambda}_j = [\lambda_{1j}, \cdots, \lambda_{kj}]^\top
            \hfill (j=1, \cdots,m)
        \]

        \item \( \displaystyle \sum_{j=1}^{m} \psi_j \mathbf{x}_j = \sum_{j=1}^{m} \psi_j \mathbf{B}\bm{\lambda}_j = \mathbf{B}\sum_{j=1}^{m} \psi_j \bm{\lambda}_j \)\\
        This means that $\{\mathbf{x_1, \cdots, x_m}\}$ are linearly independent if and only if the column vectors $\{\bm{\lambda_1, \cdots, \lambda_m}\}$ are linearly independent.

    \end{itemize}

    \item In a vector space $V$, $m$ linear combinations of $k$ vectors $\mathbf{x_1, \cdots, x_k}$ are linearly dependent if $m > k$.
\end{enumerate}


\section{Generating Set ( $\mathbb{A}$ ) \cite{mfml-1}}\label{Generating Set}
Consider a vector space $V = (\mathbb{V}, +, \cdot)$ and set of vectors $\mathbb{A} = \mathbf{\{x_1, \cdots , x_k\}} \subseteq \mathbb{V}$. If every vector $v \in V$ can be expressed as a linear combination of $\mathbf{x_1, \cdots , x_k}$, $\mathbb{A}$ is called a generating set of $V$.

\begin{itemize}
    \item Generating sets are sets of vectors that span vector (sub)spaces, i.e., every vector can be represented as a linear combination of the vectors in the generating set.
\end{itemize}


\section{Span ( $\rcmdXspan[\mathbb{A}]$ ) \cite{mfml-1}} \label{lin-alg: Span}
The set of all linear combinations of vectors in a generating set $\mathbb{A}$ is called the span of $\mathbb{A}$. If $\mathbb{A}$ spans the vector space $V$, we write $V = \rcmdXspan[\mathbb{A}]$ or $V = \rcmdXspan[\mathbf{x_1, \cdots , x_k}]$.




\section{Basis ( $\mathbb{B}$ ) \cite{mfml-1}}\label{lin-alg: Basis}
Consider a vector space $V = (\mathbb{V}, +, \cdot)$ and $\mathbb{A} \subseteq \mathbb{V}$. A generating set $\mathbb{A}$ of $V$ is called \textbf{minimal} if there exists no smaller set $\tilde{\mathbb{A}} \not\subseteq \mathbb{A} \subseteq V$ that spans $V$. Every linearly independent generating set of $V$ is minimal and is called a basis of $V$.

\vspace{0.2cm}
Let $V = (\mathbb{V}, +, \cdot)$ be a vector space and $\mathbb{B} \subseteq \mathbb{V}, \mathbb{B} \neq \varnothing$. Then, the following statements are equivalent:

\begin{itemize}
    \item $\mathbb{B}$ is a basis of $V$.
    
    \item $\mathbb{B}$ is a minimal generating set.
    
    \item $\mathbb{B}$ is a maximal linearly independent set of vectors in $V$, i.e., adding any other vector to this set will make it linearly dependent.

    \item Every vector $\mathbf{x} \in V$ is a linear combination of vectors from $\mathbb{B}$, and every linear combination is unique, i.e., with \( \displaystyle \mathbf{x} = \sum_{i=1}^{k} \lambda_i \mathbf{b}_i = \sum_{i=1}^{k} \psi_i \mathbf{b}_i \) and $\lambda_i, \psi_i \in \mathbb{R}, \mathbf{b}_i \in \mathbb{B}$, it follows that $\lambda_i = \psi_i$, $i=1,\cdots,k$

    
\end{itemize}

\vspace{0.2cm}
\textbf{Note}:
\begin{itemize}
    \item Every vector space $V$ possesses a basis $\mathbb{B}$. 
    
    \item There is no unique basis. 
    
    \item All bases possess the same number of elements, the \textbf{basis vectors}\indexlabel{basis vectors}.

    \item A basis effectively defines a \textbf{coordinate system}. (SEE: \fullref{Coordinate vector})
\end{itemize}

\subsection{Finding Basis \cite{mfml-1}}
A basis of a subspace $U = \rcmdXspan[\mathbf{x_1, \cdots, x_m}] \subseteq \mathbb{R}^n$ can be found by executing the following steps:
\begin{enumerate}
    \item Write the spanning vectors as columns of a matrix $\mathbf{A}$
    \item Determine the row-echelon form of $\mathbf{A}$
    \item The spanning vectors associated with the pivot columns are a basis of $U$
\end{enumerate}


\subsection{Ordered Basis ( $\mathit{B}$ ) \cite{mfml-1}}\label{ordered basis}
$n$-tuple an \textbf{ordered basis} of $\mathbf{V}$: $\mathit{B} = (\mathbf{b_1, \cdots, b_n})$

\subsection{Basis Change \cite{mfml-1}}\label{Basis Change}

\begin{figure}[h]
    \centering
    \includegraphics[width=\linewidth, height=2.5cm, keepaspectratio]{Pictures/maths/basis-change.png}
    \caption{Basis Change}
\end{figure}

\noindent\textbf{Notations}:
\begin{itemize}
    \item linear mapping:  $\Phi : \mathbf{V \to W}$

    \item ordered bases of $\mathbf{V}$ : \(         \mathit{B} = \mathbf{(b_1, \cdots , b_n)} \quad\quad \tilde{\mathit{B}} = \mathbf{( \tilde{b}1, \cdots , \tilde{b}_n)}     \)

    \item ordered bases of $\mathbf{W}$ : \( \mathit{C} = \mathbf{(c_1, \cdots , c_n)} \quad\quad \tilde{\mathit{C}} = \mathbf{( \tilde{c}1, \cdots , \tilde{c}_n)} \)

    \item Transformation matrix of $\Phi: \mathbf{V} \to \mathbf{W}$ wrt bases $\mathit{B}$ and $\mathit{C}$ : $\mathbf{A}_\Phi \in \mathbb{R}^{m \times n}$

    \item Transformation matrix of $\Phi: \mathbf{V} \to \mathbf{W}$ wrt bases $\tilde{\mathit{B}}$ and $\tilde{\mathit{C}}$ : $\tilde{\mathbf{A}}_\Phi \in \mathbb{R}^{m \times n}$

    \item $A_\Phi$ is \textbf{transformation matrix} of $\Phi_{CB}$ wrt ordered bases $B$, $C$\\

    \item $\Psi_{B\tilde{B}} : V \to V$ maps coordinates with respect to the new basis $\tilde{B}$ onto the (unique) coordinates with respect to the “old” basis $B$ (in $V$)

    \item $\Xi_{\tilde{C}C} : W \to W$ maps coordinates with respect to $C$ onto coordinates with respect to $\tilde{C}$

    \item  \begin{align}
        \Phi_{\tilde{C}\tilde{B}}  \notag
        &= \Xi_{\tilde{C}C} \circ \Phi_{CB} \circ \Psi_{B\tilde{B}} \\ \notag
        &= \Xi^{-1}_{C\tilde{C}} \circ \Phi_{CB} \circ \Psi_{B\tilde{B}} \\ \notag
    \end{align} 
    
\end{itemize}

\vspace{0.2cm}
\noindent \textbf{Note}:
\begin{itemize}
    \item Transformation matrices of a linear mapping $\Phi : \mathbf{V \to W}$ change if we change the bases in $\mathbf{V}$ and $\mathbf{W}$.

    \item 
\end{itemize}

\begin{theorem}[Basis Change]\label{theorem: Basis Change}
    For a linear mapping $\Phi : \mathbf{V \to W}$, ordered bases\\ 
    \(
        \mathit{B} = \mathbf{(b_1, \cdots , b_n)} \quad\quad \tilde{\mathit{B}} = \mathbf{( \tilde{b}1, \cdots , \tilde{b}_n)}  \hfill \text{ of }\mathbf{V}
    \)\\
    and\\
    \(
        \mathit{C} = \mathbf{(c_1, \cdots , c_n)} \quad\quad \tilde{\mathit{C}} = \mathbf{( \tilde{c}1, \cdots , \tilde{c}_n)}  \hfill \text{ of }\mathbf{W}
    \)\\
    and a transformation matrix $\mathbf{A}_\Phi$ of $\Phi$ with respect to $\mathit{B}$ and $\mathit{C}$, the corresponding transformation matrix $\tilde{\mathbf{A}}_\Phi$ with respect to the bases $\tilde{\mathit{B}}$ and $\tilde{\mathit{C}}$ is given as\\
    \( \tilde{\mathbf{A}}_\Phi = \mathbf{T^{-\top} \mathbf{A}_\Phi S} \)
\end{theorem}

\vspace{0.2cm}
\noindent \textbf{where}:
\begin{itemize}
    \item $\mathbf{S} \in \mathbb{R}^{n\times n}$ is the transformation matrix of $id_V$ that maps coordinates with respect to $\tilde{\mathit{B}}$ onto coordinates with respect to $\mathit{B}$

    \item $\mathbf{T} \in \mathbb{R}^{m\times m}$ is the transformation matrix of $id_W$ that maps coordinates with respect to $\tilde{\mathit{C}}$ onto coordinates with respect to $\mathit{C}$.
\end{itemize}

\vspace{0.2cm}
\noindent \textbf{Proof}:
\begin{enumerate}
    \item we can write the vectors of the new basis $\hat{\mathit{B}}$ of $\mathbf{V}$ as a linear combination of the basis vectors of $\mathit{B}$, such that\\
    \(
        \displaystyle \hat{\mathbf{b}}_j = s_{1j}\mathbf{b}_1 + \cdots + s_{nj}\mathbf{b}_n = \sum_{i=1}^{n} s_{ij}\mathbf{b}_i \hfill (j = 1,\cdots,n)
    \)\\
    and,\\
    \(
        \displaystyle \hat{\mathbf{c}}_k = t_{1k}\mathbf{c}_1 + \cdots + t_{mk}\mathbf{c}_m = \sum_{l=1}^{n} t_{lk}\mathbf{c}_l \hfill (k = 1,\cdots,m)
    \)

    \item $\mathbf{S} = ((s_{ij})) \in \mathbb{R}^{n\times n}$ as the transformation matrix that maps coordinates with respect to $\tilde{\mathit{B}}$ onto coordinates with respect to $\mathit{B}$\\
    $j$th column of $\mathbf{S}$ is the \textbf{coordinate} representation of  $\tilde{\mathbf{b}}_j$ with respect to $\mathit{B}$\\
    $\mathbf{S}$ is a \textbf{regular matrix}.
    
    
    \item $\mathbf{T} = ((T_{lk})) \in \mathbb{R}^{m\times m}$ as the transformation matrix that maps coordinates with respect to $\tilde{\mathit{C}}$ onto coordinates with respect to $\mathit{C}$\\
    $k$th column of $\mathbf{T}$ is the \textbf{coordinate} representation of $\tilde{\mathbf{c}}_k$ with respect to $\mathbf{C}$\\
    $\mathbf{T}$ is a \textbf{regular matrix}.

    \item \( \displaystyle
        \Phi(\tilde{\mathbf{b}}_j) = \sum_{k=1}^{m} \tilde{a}_{kj}\textcolor{blue}{\tilde{\mathbf{c}}_k} = \sum_{k=1}^{m} \tilde{a}_{kj} \textcolor{blue}{\sum_{l=1}^{m} t_{lk} \mathbf{c}_l} = \sum_{l=1}^{m}\left( \textcolor{Magenta}{\sum_{k=1}^{m} t_{lk}\tilde{a}_{kj}} \right)\mathbf{c}_l
    \)  \\
    \( \displaystyle
     \Phi(\textcolor{blue}{\tilde{\mathbf{b}}_j}) = \Phi \left( \textcolor{blue}{\sum_{i=1}^{m} s_{ij}\mathbf{b}_i} \right) = \sum_{i=1}^{m} s_{ij} \textcolor{red}{\Phi(\mathbf{b}_i)} = \sum_{i=1}^{m} s_{ij} \textcolor{red}{\sum_{i=1}^{m} a_{li}\mathbf{c}_l} = \sum_{l=1}^{m}\left( \textcolor{Magenta}{\sum_{i=1}^{n} a_{li}s_{ij}} \right)\mathbf{c}_l
    \)

    \item \( \displaystyle
        \sum_{k=1}^{m} t_{lk}\tilde{a}_{kj} = \sum_{i=1}^{n} a_{li}s_{ij} 
    \)\\
    \(
        \Rightarrow \mathbf{T\tilde{A}}_\Phi = \mathbf{A}_\Phi \mathbf{S} \in \mathbb{R}^{m\times n}
    \)\\
    \(
        \Rightarrow \mathbf{\tilde{A}}_\Phi = \mathbf{T^{-\top}A}_\Phi \mathbf{S}
    \)\\
    Hence, Proved!
    
\end{enumerate}








\section{Coordinate vector ( $\alpha$ ) \cite{mfml-1}}\label{Coordinate vector}

Consider a vector space $\mathbf{V}$ and an ordered basis $\mathit{B} = \mathbf{(b_1, \cdots , b_n)}$ of $\mathbf{V}$. For any $\mathbf{x \in V}$ we obtain a unique representation (linear combination) \( \mathbf{x} = \alpha_1\mathbf{b}_1 + \cdots + \alpha_n\mathbf{b}_n \) of $\mathbf{x}$ with respect to $\mathit{B}$. Then $\alpha_1, \cdots , \alpha_n$ are the \textbf{coordinates} of $\mathbf{x}$ with respect to $\mathit{B}$, and the vector \( \displaystyle \bm{\alpha} = \begin{bmatrix}
    \alpha_1 \\
    \vdots \\
    \alpha_n
\end{bmatrix} \in \mathbb{R}^n \) is the \textbf{coordinate vector}/ coordinate representation of $\mathbf{x}$ with respect to the ordered basis $\mathit{B}$.


\noindent SEE:
\begin{itemize}
    \item \fullref{ordered basis}
    \item \fullref{Linear Combination} 
\end{itemize}

\vspace{0.2cm}
\noindent \textbf{Note}:
\begin{itemize}
    \item For an $n$-dimensional vector space $\mathbf{V}$ and an ordered basis $\mathit{B}$ of $\mathbf{V}$, the mapping $\Phi : \mathbb{R}^n \to \mathbf{V}$, $\Phi(\mathbf{e}_i) = \mathbf{b}_i$, $i = 1, \cdots , n$, is linear (and because of \theoremref{theorem: isomorphic} an \textbf{isomorphism}), where $(\mathbf{e}_1, \cdots , \mathbf{e}_n)$ is the standard basis of $\mathbb{R}^n$.
\end{itemize}




\section{Rank ( $rk(\mathbf{A})$ ) \cite{mfml-1}}\label{matrix: Rank}

The number of linearly independent columns of a matrix $\mathbf{A} \in \mathbb{R}^{m \times n}$ equals the number of linearly independent rows and is called the rank rank of $\mathbf{A}$ and is denoted by $rk(\mathbf{A})$.

\begin{itemize}
    \item $rk(\mathbf{A}) = rk(\mathbf{A}^\top)$, i.e., the column rank equals the row rank.

    \item The columns of $\mathbf{A} \in \mathbb{R}^{m \times n}$ span a subspace $U \subseteq \mathbb{R}^m$ with $\dim(U) = rk(\mathbf{A})$. 

    \item The rows of $\mathbf{A} \in \mathbb{R}^{m\times n}$ span a subspace $W \subseteq \mathbb{R}^n$ with $\dim(W) = rk(\mathbf{A})$. A basis of $W$ can be found by applying Gaussian elimination to $A^\top$.

    \item For all $\mathbf{A} \in \mathbb{R}^{n\times n}$ it holds that $\mathbf{A}$ is regular (invertible) if and only if $rk(\mathbf{A}) = n$.

    \item For all $\mathbf{A} \in \mathbb{R}^{m\times n}$ and all $\mathbf{b} \in \mathbb{R}^m$ it holds that the linear equation system $\mathbf{Ax = b}$ can be solved if and only if $rk(\mathbf{A}) = rk(\mathbf{A}|\mathbf{b})$, where $\mathbf{A}|\mathbf{b}$ denotes the augmented system.

    \item For $\mathbf{A} \in \mathbb{R}^{m\times n}$ the subspace of solutions for $\mathbf{Ax = 0}$ possesses dimension $n - rk(\mathbf{A})$. This subspace is called the \textbf{kernel space}\indexlabel{kernel space} or the \textbf{null space}\indexlabel{null space}.

    \item A matrix $\mathbf{A} \in \mathbb{R}^{m\times n}$ has \textbf{full rank}\indexlabel{matrix: full rank} if its rank equals the largest possible rank for a matrix of the same dimensions. This means that the rank of a full-rank matrix is the lesser of the number of rows and columns, i.e., $rk(A) = \min(m, n)$. A matrix is said to be \textbf{rank deficient}\indexlabel{matrix: rank deficient} if it does not have full rank.

\end{itemize}


\section{Linear Mappings/ vector space homomorphism/ linear transformation ( $\Phi(\mathbf{x})$ ) \cite{mfml-1}}\label{Linear Mappings/ vector space homomorphism/ linear transformation}

\begin{figure}[h]
    \centering
    \includegraphics[width=\linewidth, height=2.5cm, keepaspectratio]{Pictures/maths/linear-mapping.png}
    \caption{Linear Mappings/ vector space homomorphism/ linear transformation}
\end{figure}

For vector spaces $V$, $W$, a mapping $\Phi : V \to W$ is called a linear mapping (or vector space homomorphism/ linear transformation) if:
\[
    \forall \mathbf{x, y} \in V, \forall\lambda, \psi \in \mathbb{R} : \Phi(\lambda\mathbf{x} + \psi\mathbf{y}) = \lambda\Phi(\mathbf{x}) + \psi\Phi(\mathbf{y})
\]

\begin{itemize}
    \item we can represent linear mappings as matrices
    \item Consider a mapping $\Phi : \mathbb{V} \to \mathbb{W}$, where $\mathbb{V}$, $\mathbb{W}$ can be arbitrary sets. Then $\Phi$ is called
    \begin{itemize}
        \item \textbf{Injective} if $\forall \mathbf{x, y} \in \mathbb{V}: \Phi(\mathbf{x}) = \Phi(\mathbf{y}) \Rightarrow \mathbf{x} = \mathbf{y}$ \indexlabel{Injective mapping}

        \item \textbf{Surjective} if $\Phi(\mathbb{V}) = \mathbb{W}$\indexlabel{Surjective mapping}\\
        If $\Phi$ is surjective, then every element in $\mathbb{W}$ can be “\textbf{reached}” from $\mathbb{V}$ using $\Phi$.

        \item \textbf{Bijective} if it is injective and surjective.\indexlabel{Bijective mapping}\\
        A bijective $\Phi$ can be “\textbf{undone}”, i.e., there exists a mapping $\Psi : \mathbb{W} \to \mathbb{V}$ so that $\Psi \circ \Phi(\textbf{x}) = \textbf{x}$. This mapping $\Psi$ is then called the \textbf{inverse} of $\Phi$ and normally denoted by $\Phi^{-1}$.\indexlabel{inverse linear mapping}
        
    \end{itemize}

    \item \textbf{Isomorphism}\indexlabel{Isomorphism}: $\Phi : V \to W$ linear and bijective\\
    
    \begin{theorem}\label{theorem: isomorphic}
        Finite-dimensional vector spaces $V$ and $W$ are isomorphic if and only if $\dim(V) = \dim(W)$.
    \end{theorem}
    
    Intuitively, this means that vector spaces of the same dimension are kind of the same thing, as they can be transformed into each other without incurring any loss.\\
    This gives the justification to treat $\mathbb{R}^{m\times n}$ (the vector space of ${m\times n}$-matrices) and $\mathbb{R}^{mn}$ (the vector space of vectors of length $mn$) the same, as their dimensions are $mn$, and there exists a linear, bijective mapping that transforms one into the other.

    \item \textbf{Endomorphism}\indexlabel{Endomorphism}: $\Phi : V \to V$ linear

    \item \textbf{Automorphism}\indexlabel{Automorphism}: $\Phi : V \to V$ linear and bijective

    \item We define $id_V : V \to V , \mathbf{x} \to \mathbf{x}$ as the \textbf{identity mapping}\indexlabel{identity mapping} or \textbf{identity automorphism}\indexlabel{identity automorphism} in V.

    \item Consider vector spaces $V, W, X$. Then:
    \begin{itemize}
        \item For linear mappings $\Phi : V \to W$ and $\Psi : W \to X$, the mapping $\Psi \circ \Phi : V \to X$ is also linear.

        \item If $\Phi : V \to W$ is an isomorphism, then $\Phi^{-1} : W \to V$ is an isomorphism, too.

        \item If $\Phi : V \to W$, $\Psi : V \to W$ are linear, then $\Phi + \Psi$ and $\lambda\Phi$, $\lambda \in \mathbb{R}$, are linear, too.
    \end{itemize}
    
\end{itemize}

\subsection{Matrix Representation of Linear Mappings \cite{mfml-1}}\label{Matrix Representation of Linear Mappings}

\begin{itemize}
    \item Any $n$-dimensional vector space is \textbf{isomorphic} to $R^n$. (SEE: \theoremref{theorem: isomorphic})
    
\end{itemize}

\subsection{Transformation Matrix ( $\mathbf{A}_\Phi$ ) \cite{mfml-1}}\label{Transformation Matrix}

\begin{itemize}
    \item Consider vector spaces $\mathbf{V}$, $\mathbf{W}$ with corresponding (ordered) bases $\mathit{B} = \mathbf{(b_1, \cdots , b_n)}$ and $\mathit{C} = \mathbf{(c_1, \cdots , c_m)}$. (SEE: \fullref{ordered basis})

    \item Moreover, we consider a linear mapping $\Phi : \mathbf{V} \to \mathbf{W}$. 

    \item For $j \in \{1, \cdots , n\}$, \(     \displaystyle \Phi(\mathbf{b}_j) = \alpha_{1j}\mathbf{c}_1 + \cdots + \alpha_{mj}\mathbf{c}_m = \sum_{i=1}^{m} \alpha_{ij}\mathbf{c}_{i} \) is the \textbf{unique} representation of $\Phi(\mathbf{b}_j)$ with respect to $\mathit{C}$. 

    \item The $m \times n$-matrix $\mathbf{A}_\Phi$, whose elements are given by \( \mathbf{A}_\Phi(i, j) = \alpha_{ij} \), is called \textbf{transformation matrix} of $\Phi$ (with respect to the ordered bases $\mathit{B}$ of $\mathbf{V}$ and $\mathit{C}$ of $\mathbf{W}$)

    \item The coordinates of $\Phi(\mathbf{b}_j)$ with respect to the ordered basis $\mathit{C}$ of $\mathbf{W}$ are the $j$-th column of $\mathbf{A}_\Phi$. (SEE: \fullref{Coordinate vector})

    \item Consider (\textbf{finite-dimensional}) vector spaces $\mathbf{V, W}$ with ordered bases $\mathit{B, C}$ and a linear mapping $\Phi : \mathbf{V \to W}$ with transformation matrix $\mathbf{A}_\Phi$. \\
    If $\hat{\mathbf{x}}$ is the coordinate vector of $\mathbf{x} \in \mathbf{V}$ with respect to $\mathit{B}$ and $\hat{\mathbf{y}}$ the coordinate vector of $\mathbf{y} = \Phi(\mathbf{x}) \in \mathbf{W}$ with respect to $\mathit{C}$, then $\hat{\mathbf{y}} = \mathbf{A}_\Phi\hat{\mathbf{x}}$.

    \item The transformation matrix can be used to map coordinates with respect to an ordered basis in $\mathbf{V}$ to coordinates with respect to an ordered basis in $\mathbf{W}$.
\end{itemize}



\vspace{4cm}
\url{https://drive.google.com/file/d/1JS4tlbMRYv8oHYrLgISR_NV9y6-_2j5h/edit}\\
61 / 417











































































































































































