\chapter{Multivariate Distributions \cite{ism-1,mfml-1}} \label{Multivariate Distributions}

\begin{enumerate}
     \item if all distribution functions are identical, $F_{X_1}(x)=\cdots=F_{X_K}(x)=F(x)$, for all $x$, then $X_1, X_2,\cdots,X_K$ are i.i.d. with distribution function $F$.
\end{enumerate}

\section{Joint PDF/ PMF ($f_{XY}(x, y)$) \cite{ism-1}}\label{Multivariate Distributions: Joint PDF/ PMF}

\[
    f_{XY}(x, y) 
    = Pr(X = x, Y = y)
    \begin{cases}
        \in \mathbb{R} & \text{ (Discrete (PMF))}\\
        = 0 \text{ (always)} & \text{ (Continuous (PDF))}
    \end{cases}
\]

\section{Marginal PMF/ PDF ($f_X(x)$ / $f_Y(y)$) \cite{ism-1,mfml-1}}\label{Multivariate Distributions: Marginal PMF/ PDF}

\begin{table}[H]
    \begin{minipage}{0.49\linewidth}
        \[
            f_X(x)
            = \begin{cases}
                \dsum_{y=0}^\infty f_{XY}(x,y) & \text{ (Discrete)}\\[2ex]
                \dint_{-\infty}^\infty f_{XY}(x,y) dy & \text{ (Continuous)}\\
            \end{cases}
        \]
    \end{minipage}
    \hfill
    \begin{minipage}{0.49\linewidth}
        \[
            f_Y(y)
            = \begin{cases}
                \dsum_{x=0}^\infty f_{XY}(x,y) & \text{ (Discrete)}\\[2ex]
                \dint_{-\infty}^\infty f_{XY}(x,y) dx & \text{ (Continuous)}\\
            \end{cases}
        \]
    \end{minipage}
\end{table}

\begin{enumerate}
    \item if $x = [x_1, \cdots , x_D]^\top$, we obtain the marginal:\cite{mfml-1}
    \[
        p(x_i) = \dint
        p(x_1, \cdots , x_D)dx_{\backslash i}
    \]
    by repeated application of the sum rule where we integrate/ sum out all random variables except $x_i$, which is indicated by $\backslash i$, which reads “all except $i$”.

\end{enumerate}

\section{Joint CDF ($F_{XY}(x, y)$) \cite{ism-1,mfml-1}} \label{Multivariate Distributions: joint CDF}

\[
    F_{XY}(x, y)
    = Pr(X \leq x, Y \leq y) 
    = \begin{cases}
        \dsum_{k=0}^{x}
        \dsum_{l=0}^{y} f_{XY}(k,l) & \text{ (discrete)}\\[3ex]
        \dint_{-\infty}^{x}
        \dint_{-\infty}^{y} 
        f_{XY}(u,v)dudv & \text{ (continuous)}\\
    \end{cases}
\]
\[
    Pr((X,Y)\in A) 
    = \displaystyle\iint_A f_{XY}(u,v) dudv
    \hfill
    (A \subset \mathbb{R}^2)
\]

\begin{enumerate}
    \item for $2$ variables, aka \textbf{bivariate distribution function}\indexlabel{bivariate distribution function}

    \item in general, multivariate distribution function
    \[
        F_{X_1,X_2,\cdots,X_K}(x_1,x_2,\cdots,x_K)
        = Pr(X_1 \leq x_1, X_2 \leq x_2,\cdots,X_K \leq x_K)
        \hfill
        \text{\cite{ism-1}}
    \]

    \item joint distribution function contains all the information on how the random variables are related to each other

    \item A cumulative distribution function (cdf) of a multivariate real-valued random variable $X$ with states $x \in R^D$ is given by: \cite{mfml-1}
    \[
        F_X(x) = P(X_1 \leq x_1, \cdots , X_D \leq x_D)
        \hfill
        \text{\cite{mfml-1}}
    \]
    \[
        \displaystyle
        F_X(x) =
        \int_{-\infty}^{x_1}
        \cdots
        \int_{-\infty}^{x_D}
        f(z_1,\cdots,z_D) dz_1,\cdots,dz_D
    \]
\end{enumerate}

\section{Marginal CDF \cite{ism-1}}\label{Multivariate Distributions: Marginal CDF}
\[
    \hfill
    F_X(x) = \lim_{y\to\infty} F_{XY}(x, y)
    \hfill
    F_Y(y) = \lim_{x\to\infty} F_{XY}(x, y)
    \hfill
\]


\section{Conditional PMF of $X$ given $Y = y$ ($f_{X|Y}(x|y)$) \cite{ism-1}}\label{Multivariate Distributions: conditional PMF}

\[
    f_{X|Y}(x|y) = Pr(X=x|Y=y)
    =\dfrac{Pr(X=x, Y=y)}{Pr(Y=y)} 
    = \dfrac{f_{XY}(x,y)}{f_Y(y)}
    \hfill
    (f_Y(y) > 0)
\]

\section{Independence of random variables \cite{ism-1}} \label{Multivariate Distributions: Independence of random variables}

\begin{enumerate}
    \item Two random variables $X$ and $Y$ are called \textbf{independent} when the bivariate distribution function is equal to the product of the marginal distribution functions: $F_{XY}(x, y) = F_X(x)F_Y(y)$

    \item When the random variables X and Y are \textbf{independent}, the conditional PMF becomes equal to the marginal PMF: $f_{XY}(x, y) = f_X(x)f_Y(y) \quad\forall (x,y) \in \mathbb{R}^2$

    \item If $X$ and $Y$ are independent, then $f_{X|Y}(x|y) = f_X(x)$

    \item The random variables $X_1, X_2,\cdots, X_K$ are called mutually independent when the joint distribution function is the product of the marginal distribution functions:
    \[
        F_{X_1\cdots X_K}(x_1,\cdots, x_K) = F_{X_1}(x)\cdots F_{X_K}(x_K)
        \hfill
        (\forall x_k \in \mathbb{R})
    \]

    \item Two random variables X, Y are statistically independent if and only if $p(x,y) = p(x)p(y)$ \cite{mfml-1}
    \begin{enumerate}
        \item $ p(y | x) = p(y)$
        \item $ p(x | y) = p(x)$
    \end{enumerate}

    \item Intuitively, two random variables $X$ and $Y$ are independent if the value of $y$ (once known) does not add any additional information about $x$ (and vice versa) \cite{mfml-1}
\end{enumerate}

\section{Conditional Independence \cite{mfml-1}} \label{Multivariate Distributions: Conditional Independence}

\begin{enumerate}
    \item Two random variables $X$ and $Y$ are conditionally independent given $Z$ if and only if for all $z \in Z$
    \begin{enumerate}
        \item $p(x, y | z) = p(x | z)p(y | z)$
        \item alternatively, $p(x | y, z) = p(x | z)$
    \end{enumerate}

    where $Z$ is the set of states of random variable $Z$. We write $X \doubleuptack Y | Z$ to denote that $X$ is conditionally independent of $Y$ given $Z$

    \item interpretation:
    \begin{enumerate}
        \item "given knowledge about $z$, the distribution of $x$ and $y$ factorizes"

        \item (alternatively) “given that we know $z$, knowledge about $y$ does not change our knowledge of $x$”
    \end{enumerate}

    \item Independence can be cast as a special case of conditional independence if we write $X \doubleuptack Y | \varnothing$
\end{enumerate}


\section{Conditional expectation \cite{ism-1}} \label{Multivariate Distributions: Conditional expectation}

\begin{enumerate}
    \item[] $X, Y$ : Random variables
    
    \item[] $\psi$ : any function

    \item conditional expectation of $\psi(Y)$ given $X = x$:
    \[
        \mathbb{E}(\psi(Y)|X=x)
        = \dsum_{y=0}^{\infty}
        \psi(y)f_{Y|X}(y|x)
    \]

    this expectation is thus a function of $x$

    \item $
        \psi(y) = y
        \Rightarrow
        \mu_Y(x) = \mathbb{E}(Y|X = x)
        \text{ (or) }
        \mu_Y(x) = \mathbb{E}(Y|X = X)
    $
\end{enumerate}


\section{Expected Value \cite{ism-1,mfml-1}}\label{Multivariate Distributions: Expected Value}
\[
\begin{aligned}
    \mathbb{E}(\mu_Y(X))
    &= \dsum_{x=0}^{\infty} \mu_Y(x)f_X(x)
    &= \dsum_{x=0}^{\infty} 
        \mathbb{E}(Y|X=x)f_X(x) \\
    &= \dsum_{x=0}^{\infty}
        \dsum_{y=0}^{\infty} yf_{XY}(x,y)
    &= \dsum_{x=0}^{\infty}
        \dsum_{y=0}^{\infty} yf_{Y|X}(y|x)f_X(x) \\
    &= \dsum_{y=0}^{\infty} yf_Y(y) = \mu_Y
    & \text{\cite{ism-1}}
\end{aligned}
\]


\begin{enumerate}
    \item The expected value of a function $g : R \to R$ of a univariate continuous random variable $X \sim p(x)$ is given by $
        \mathbb{E}_X[g(x)] = 
        \begin{cases}
            \dint_X g(x)p(x)dx \\[2ex]
            \dsum_{x\in X} g(x)p(x)
        \end{cases}
    $ where $X$ is the set of possible outcomes (the target space) of the random variable $X$

    \item The expected value of a function of a random variable is sometimes referred to as the \textbf{law of the unconscious statistician}\indexlabel{law of the unconscious statistician}

    \item We consider multivariate random variables $X$ as a finite vector of univariate random variables $[X_1, \cdots , X_D]^\top$. For multivariate random variables, we define the expected value element wise:
    \[
        \mathbb{E}_X[g(x)]
        = \begin{bmatrix}
            \mathbb{E}_{X_1}[g(x)] \\
            \vdots \\
            \mathbb{E}_{X_D}[g(x)] \\
        \end{bmatrix} \in \mathbb{R}^D
    \]
    $\mathbb{E}_{X_d}$ indicates that we are taking the expected value with respect to the $d$th element of the vector $x$

    \item The mean of a random variable $X$ with states $x \in R^D$ is an average and is defined as: \cite{mfml-1}
    \[
        \mathbb{E}_X[x]
        = \begin{bmatrix}
            \mathbb{E}_{X_1}[x_1] \\
            \vdots \\
            \mathbb{E}_{X_D}[x_D] \\
        \end{bmatrix} \in \mathbb{R}^D
    \]
    \[
        \mathbb{E}_{X_d}[x_d]
        = \begin{cases}
            \dint_X x_d p(x_d) dx_d
            & \text{if $X$ is a continuous random variable}\\[2ex]
            \dsum_{x_i \in X} x_i p(x_d=x_i)
            & \text{if $X$ is a discrete random variable}
        \end{cases}
    \]

    \item Consider a random variable $X$ with mean $\mu$ and covariance matrix $\Sigma$ and a (deterministic) affine transformation $y = Ax + b$ of $x$. Then:
    \[
        \mathbb{E}_Y[y] 
        = \mathbb{E}_X[Ax + b] 
        = A\mathbb{E}_X[x] + b 
        = A\mu + b
    \]
\end{enumerate}

\section{Empirical Mean/ Sample Mean \cite{mfml-1}} \label{Multivariate Distributions: Empirical Mean/ Sample Mean}

\begin{enumerate}
    \item we have a \textbf{finite dataset} (of size $N$)

    \item an empirical statistic that is a function of a finite number of identical random variables, $X_1, \cdots , X_N$ with realizations $x_1, \cdots , x_N$

    \item The empirical mean vector is the arithmetic average of the observations for each variable, and it is defined as:
    \[
        \bar{x} = \dfrac{1}{N}
        \dsum_{n=1}^{N} x_n
        \hfill
        (x_n \in \mathbb{R}^D)
    \]

    \item $\mathbb{E}[x \pm y] = \mathbb{E}[x] \pm \mathbb{E}[y]$
\end{enumerate}


\section{Conditional Variance \cite{ism-1}} \label{Multivariate Distributions: Conditional Variance}

\begin{enumerate}
    \item[] $g(x, y)$ : general function of x and y

    \item[] $
        VAR(Y|X=x) 
        = \mathbb{E}((Y-\mu_y(x))^2|X=x)
        = \dsum_{y=0}^\infty
        (y-\mu_y(x))^2 f_{Y|X}(y|x)
    $

    \item we calculate the variance around the conditional mean $\mu _Y(x)$, and not around $\mu _Y$, since $\mu _Y(x)$ is the expected value for Y when we condition on or know that $X = x$

    \item conditional variance is also a function of $x$, and we may denote it by $\sigma _Y^2(x)$
    \begin{enumerate}
        \item[] $VAR(Y|X=x) = \sigma _Y^2(x)$

    \end{enumerate}

    \item $VAR(Y) = \mathbb{E}[VAR(Y|X = X)] + VAR(\mathbb{E}(Y|X = X))$

    \item $
        \mathbb{E}[g(X,Y)]
        =\dsum_{x=0}^{\infty}
            \dsum_{y=0}^{\infty}
            g(x,y) f_{XY}(x,y)
    $

    \item if $X$ and $Y$ are independent then $\mathbb{E}(XY) = \mathbb{E}(X)\mathbb{E}(Y)$
\end{enumerate}

\[
\begin{aligned}
    \mathbb{E}[\sigma^2_Y(X)]
    &= \dsum_{x=0}^{\infty}
        \sigma_Y^2(x)f_X(x)\\
    &= \dsum_{x=0}^{\infty}
        \dsum_{y=0}^{\infty}
        (y - \mu_Y (x))^2 f_{Y |X} (y|x) f_X (x)\\
    &= \dsum_{x=0}^{\infty}
        \dsum_{y=0}^{\infty}
        [(y - \mu_Y ) ^2 + 2(y - \mu_Y )(\mu_Y - \mu_Y (x)) + (\mu_Y - \mu_Y (x))^2] f_{XY} (x, y)\\
    &= \dsum_{y=0}^{\infty} 
        (y - \mu_Y )^2 f_Y (y) 
        - \dsum_{x=0}^{\infty} 
        (\mu_Y (x) - \mu_Y ) ^2 f_X (x)\\
    &= \sigma_Y^2 - VAR(\mu_Y(X))
\end{aligned}
\]

\section{Covariances/ Cross-covariance \cite{ism-1}} \label{Multivariate Distributions: Covariances/ Cross-covariance}

\begin{enumerate}
    \item $
        COV(X, Y) 
        = \sigma_{XY} 
        = \mathbb{E}[(X - \mathbb{E}(X))(Y - \mathbb{E}(Y))] 
        = \mathbb{E}(XY) - (\mathbb{E}(X))(\mathbb{E}(Y))
    $ \hfill \cite{ism-1}

    \item $
        Cov_{X,Y}[x, y] 
        := \mathbb{E}_{X,Y}[(x - \mathbb{E}_X[x])(y - \mathbb{E}_Y[y])]
    $ \hfill (Univariate) \cite{mfml-1}

    \item $
        Cov[x, y] 
        = \mathbb{E}[xy^\top] - \mathbb{E}[x]\mathbb{E}[y]^\top 
        = Cov[y, x]^\top 
        \in R^{D\times E}
    $ \hfill (Multivariate) \cite{mfml-1}

    

\end{enumerate}

\subsection*{Properties}
\begin{enumerate}[itemsep=0.2cm]
    \item $COV(X, X) = VAR(X) = \sigma_X^2$ 
    \hfill \cite{ism-1}

    \item $Cov[x, x] = V_X[x]$ 
    \hfill \cite{mfml-1}

    \item $COV(X, Y) = COV(Y, X)$ 
    \hfill \cite{ism-1}

    \item $
        COV(\alpha X, Y) = \alpha \cdot COV(X, Y) 
        \quad \forall \alpha \in \mathbb{R}
    $ 
    \hfill \cite{ism-1}

    \item $
        COV(X + \alpha, Y) = COV(X, Y) 
        \quad \forall \alpha \in \mathbb{R}
    $ 
    \hfill \cite{ism-1}

    \item $COV(X + Y, Z) = COV(X, Z) + COV(Y, Z)$ 
    \hfill \cite{ism-1}

    \item $VAR(X \pm Y) = VAR(X) \pm 2COV(X, Y) + VAR(Y)$ 
    \hfill \cite{ism-1}

    \item $V[x \pm y] = V[x] + V[y] \pm Cov[x, y] \pm Cov[y, x]$ 
    \hfill \cite{mfml-1}

    \item $
        (X_i, Y_i) \sim F_{XY}
        \Rightarrow 
        COV(X_i, Y_i) = COV(X, Y)$ 
    \hfill \cite{ism-1}

    \item $
        \mathbb{E}[(X_i - \mu_X)(Y_j - \mu_Y)] 
        = \mathbb{E}(X_j - \mu_X)\mathbb{E}(Y_i - \mu_Y) 
        = 0  \text{, when } i \neq j
    $
    \hfill \cite{ism-1}

    \item If $X$ and $Y$ are \textbf{independent}:
    \begin{enumerate}
        \item $COV(X, Y) = 0$ \hfill \cite{ism-1}
        
        \item $Cov_{X,Y}[x, y] = 0$ \hfill \cite{mfml-1}

        \item may not hold in converse, i.e., two random variables can have covariance zero but are not statistically independent. To understand why, recall that covariance measures only linear dependence. Therefore, random variables that are non-linearly dependent could have covariance zero. \cite{mfml-1}
        
    \end{enumerate}

    \item the covariance is affected by transformations
    
    \item $V_X$ is variance

    \item indicates how two random variables are related

\end{enumerate}

\section{Empirical Covariance ($\Sigma$) \cite{mfml-1}}\label{Multivariate Distributions: Empirical Covariance}

the empirical covariance matrix is a $D\times D$ matrix:
\[
    \Sigma
    := \dfrac{1}{N}
    \dsum_{n=1}^{N}
    (x_n - \bar{x})(x_n - \bar{x})^\top
    \hfill
    \text{(biased estimate)}
\]
\[
    \Sigma
    := \dfrac{1}{N-1}
    \dsum_{n=1}^{N}
    (x_n - \bar{x})(x_n - \bar{x})^\top
    \hfill
    \text{(unbiased/ corrected estimate)}
\]


\section{Variance \cite{mfml-1}}\label{Multivariate Distributions: Variance}

The variance of a random variable $X$ with states $x \in R^D$ and a mean vector $\mu \in R^D$ is defined as
\[
    \mu = \mathbb{E}_X(x)
\]

\[
\begin{aligned}
    \mathbb{V}_X[x] 
    &= Cov_X[x,x]\\
    &= \mathbb{E}_X[(x-\mu)(x-\mu)^\top]
    = \mathbb{E}_X[xx^\top]
    = \mathbb{E}_X[x]\mathbb{E}_X[x]^\top\\
    &= \begin{bmatrix}
        Cov[x_1,x_1] & Cov[x_1,x_2] & \cdots & Cov[x_1,x_D]\\
        Cov[x_2,x_1] & Cov[x_2,x_2] & \cdots & Cov[x_2,x_D]\\
        \vdots & \vdots & \ddots & \vdots \\
        Cov[x_D,x_1] & Cov[x_D,x_2] & \cdots & Cov[x_D,x_D]\\
    \end{bmatrix}
\end{aligned}
\]

\begin{enumerate}[itemsep=0.2cm]
    \item $\mathbb{V}_X[x] := \mathbb{E}_X[(x - \mu)^2]$

    \item $\mathbb{V}_X[x] = \mathbb{E}_X[x^2] - (\mathbb{E}X[x])^2$ \hfill (\textbf{raw-score formula for variance}\indexlabel{raw-score formula for variance})

    \item $
        \dfrac{1}{N^2}
        \dsum_{i,j=1}^{N} (x_i - x_j)^2
        = 2 \left[ 
            \dfrac{1}{N}\dsum_{i=1}^{N} x_i^2
            -
            \left( 
                \dfrac{1}{N}\dsum_{i=1}^{N} x_i 
            \right)^2
        \right]
    $

    \item we can express the sum of pairwise distances (of which there are $N^2$ of them) as a sum of deviations from the mean (of which there are $N$)

    \item The $D\times D$ matrix is called the \textbf{covariance matrix}\indexlabel{covariance matrix} of the multivariate random variable $X$

    \item The covariance matrix is symmetric and positive semidefinite and tells us something about the spread of the data

    \item The off-diagonal entries are the cross-covariance terms $Cov[x_i, x_j]$ for $i, j = 1, \cdots , D, i \neq j$

    \item Consider a random variable $X$ with mean $\mu$  and covariance matrix $\Sigma$ ($= \mathbb{E}[xx^\top] - \mu \mu ^\top$) and a (deterministic) affine transformation $y = Ax + b$ of $x$. Then:
    \begin{enumerate}
        \item $
            \mathbb{V}_Y[y] 
            = \mathbb{V}_X[Ax + b] 
            = \mathbb{V}_X[Ax] 
            = A\mathbb{V}_X[x]A^\top = A\Sigma A^\top
        $

        \item 
        $\begin{aligned}
            Cov[x, y] 
            &= \mathbb{E}[x(Ax + b)^\top ] - \mathbb{E}[x]\mathbb{E}[Ax + b]^\top \\
            &= \mathbb{E}[x]b^\top  + \mathbb{E}[xx^\top ]A^\top  - \mu b^\top  - \mu \mu ^\top A^\top \\
            &= \mu b^\top  - \mu b^\top  + (\mathbb{E}[xx^\top ] - \mu \mu ^\top )A =  \Sigma A^\top
        \end{aligned}$

    \end{enumerate}

    \item If $X$, $Y$ are (statistically) \textbf{independent}, then: $V_{X,Y}[x + y] = V_X[x] + V_Y[y]$

    \item (\textbf{law of total variance}\indexlabel{law of total variance}) $V_X[x] = E_Y[V_X[x|y]]+ V_Y[E_X[x|y]]$ \\
    the (total) variance of $X$ is the expected conditional variance plus the variance of a conditional mean
\end{enumerate}

\section{Sample Covariance \cite{ism-1}}\label{Multivariate Distributions: Sample Covariance}

\begin{enumerate}
    \item[] $(X_i, Y_i) \sim F_{XY}$

    \item[] $\bar{X}$: sample average of $X$

    \item[] $\bar{Y}$: sample average of $Y$

    \item $
        S_{XY} = \dfrac{1}{n-1}
        \dsum_{i=1}^{n}
        (X_i - \bar{X})(Y_i - \bar{Y})
    $

    \item $\begin{aligned}
        \mathbb{E}(S_{XY})
        &= \dfrac{1}{n-1}
        \dsum_{i=1}^{n}
        \mathbb{E}[(X_i - \bar{X})(Y_i - \bar{Y})] \\
        &= \dfrac{1}{n-1}
        \dsum_{i=1}^{n}
        \mathbb{E}[(X_i -\mu_X + \mu_X - \bar{X})
        (Y_i -\mu_Y + \mu_Y - \bar{Y})] \\
        &= \dfrac{1}{n-1}
        \dsum_{i=1}^{n}
        E[(X_i - \mu_X )(Y_i - \mu_Y ) - (X_i - \mu_X )(\bar{Y} - \mu_Y ) \\
        & \quad\quad\quad\quad - (\bar{X} - \mu_X )(Y_i - \mu_Y ) + (\bar{X} - \mu_X )(\bar{Y} - \mu_Y )] \\
        &= \dfrac{n}{n-1}COV(X,Y) - \dfrac{2}{n(n-1)}
        \dsum_{i=1}^n E[(X_i - \mu_X )(Y_i - \mu_Y )] \\
        &\quad\quad\quad\quad + \dfrac{1}{n(n-1)} \dsum_{i=1}^n
        E[(X_i - \mu_X )(Y_i - \mu_Y )]\\
        &=COV(X, Y )
    \end{aligned}$
\end{enumerate}


\section{Maximum Likelihood Estimation \cite{ism-1}} \label{Multivariate Distributions: Maximum Likelihood Estimation}

\begin{enumerate}[itemsep=0.2cm]
    \item PDF/ PMF: $f_\theta$

    \item If $X_1, X_2,\cdots, X_n$ are i.i.d. with density $f_\theta$ , the likelihood function is given by $L (\theta|X_1, X_2,\cdots, X_n) = \dsum ^n _{i=1} f_\theta (X_i)$ and the log likelihood function is given by
    \[
        l_\theta 
        \equiv  (\theta|X_1, X_2,\cdots, X_n) 
        = \dsum_{i=1}^n \log (f_\theta (X_i))        
    \]

    \item The maximum likelihood estimator $\hat{\theta} = (\hat{\theta}_1, \hat{\theta}_2,\cdots, \hat{\theta}_n)^\top$ is the set of parameters that maximizes the log likelihood function.

    \item It can often be determined by solving the set of equations given by 
    \[
        \dfrac{\partial l_\theta}{\partial \theta_k}
        = \dsum_{i=1}^n \dSquareBrac{
            \dParenBrac{
                \dfrac{\partial f_\theta(X_i)}{\partial \theta_k}
            } /
            f_\theta(X_i)
        }
        = 0
    \]
\end{enumerate}



\section{Pearson’s Correlation Coefficient ($\rho_P$) \cite{ism-1,mfml-1}} \label{Multivariate Distributions: Pearson’s Correlation Coefficient}

\begin{enumerate}
    \item[] $X,Y$: random variables
    
    \vspace{0.1cm}
    \item[] $
        Z_X = \dfrac{X - \mu_X}{\sigma_X},
        Z_Y = \dfrac{Y - \mu_Y}{\sigma_Y}
    $ : standardized random variables

    \vspace{0.2cm}
    
    \item[] 
    \[
        \rho_P 
        = CORR(X,Y)
        = COV(Z_X, Z_Y)
        = \mathbb{E}\dParenBrac{ \dfrac{X - \mu_X}{\sigma_X}} \dParenBrac{ \dfrac{Y - \mu_Y}{\sigma_Y} }
        = \dfrac{COV(X,Y)}{\sqrt{VAR(X)} \sqrt{VAR(Y)}}
        \hfill \text{\cite{ism-1}}
    \]
    
    \item[] $
        corr(x,y)
        = \dfrac{Cov[x,y]}{\sqrt{\mathbb{V}(x) \mathbb{V}(y)}}
    $ \hfill \cite{mfml-1}
    
    \vspace{0.2cm}
    
    \item Pearson’s correlation coefficient is a way of quantifying how two variables “co-relate”

    \item If Pearson’s correlation is positive, the two variables $X$ and $Y$ move in the same direction. If $X$ is increasing then $Y$ should be increasing as well

    \item If Pearson’s correlation is negative, the random variables move in opposite directions.

    \item When Pearson’s correlation coefficient is zero the random variables are called uncorrelated.

    \item Pearson’s correlation coefficient also measures the concordance or discordance in a certain way

    \item fits very well with the \textbf{bivariate normal distribution}

    \item Pearson’s product-moment estimator is also \textbf{sensitive to outliers}

    \item $1-\rho_P$ is not a distance measure, but $\sqrt{1-\rho_P}$ is a distance measure

    \item The correlation matrix is the covariance matrix of standardized random variables, $x/\sigma(x)$

    \item indicates how two random variables are \textbf{related}

    \item Positive correlation $corr[x, y]$ means that when $x$ grows, then $y$ is also expected to grow. 

    \item Negative correlation means that as $x$ increases, then $y$ decreases.

\end{enumerate}

\subsection*{Properties \cite{ism-1}}

\begin{enumerate}
    \item $CORR(X, Y) \in [-1, 1]$

    \item If $CORR(X, Y) = 1$, then $Y = aX + b$, where $a > 0$

    \item If $CORR(X, Y) = -1$, then $Y = aX + b$, where $a < 0$

    \item $CORR(aX + b, cY + d) = CORR(X, Y)$  for $a, c > 0$ or $a, c < 0$

    
\end{enumerate}

\begin{customTableWrapper}{1}
\begin{table}[H]
    \centering
    \begin{tabular}{l l l l l l} 
        0.90 & < & $\dabs{\rho_P}$ & $\leq$ & 1.00 & \textbf{Very strong} correlation\\
        0.70 & < & $\dabs{\rho_P}$ & $\leq$ & 0.90  & \textbf{Strong} correlation\\
        0.50 & < & $\dabs{\rho_P}$ & $\leq$ & 0.70 &  \textbf{Moderate} correlation\\
        0.30 & < & $\dabs{\rho_P}$ & $\leq$ & 0.50  & \textbf{Low} correlation\\
        0  &  $\leq$ & $\dabs{\rho_P}$ & $\leq$ & 0.30 &  \textbf{Negligible} correlation\\
    \end{tabular}
    \caption{Inferring Pearson’s Correlation Coefficient values \cite{ism-1}}
\end{table}
\end{customTableWrapper}


\section{Estimate for Product-Moment/ Sample correlation coefficient ($r_P$) \cite{ism-1}} \label{Multivariate Distributions: Estimate for Product-Moment/ Sample correlation coefficient}

\begin{enumerate}
    \item[] $S_X^2 , S_Y^2$: sample variances
    \item[] $r_P$: estimator

    \[
    \begin{aligned}
        r_P
        &= \dfrac{S_{XY}}{S_X S_Y}
        = \dfrac{
            \dsum_{i=1}^{n} (X_i - \bar{X})(Y_i - \bar{Y})
        }{(n-1)S_X S_Y}
        = \dfrac{
            \dsum_{i=1}^{n} (X_i - \bar{X})(Y_i - \bar{Y})
        }{\sqrt{
            \dsum_{i=1}^{n} (X_i - \bar{X})^2
            \dsum_{i=1}^{n} (Y_i - \bar{Y})^2
        }}\\
        &= \dfrac{
            \dsum_{i=1}^{n} X_iY_i - 
            n\bar{X}\bar{Y}
        }{(n-1)S_X S_Y}
        = \dfrac{
            n\dsum_{i=1}^n X_iY_i - \dsum_{i=1}^n X_i \dsum_{i=1}^n Y_i
        }{\sqrt{
            n\dsum_{i=1}^{n} X_i^2 - \dParenBrac{\dsum_{i=1}^{n} X_i}^2
        }\sqrt{
            n\dsum_{i=1}^{n} Y_i^2 - \dParenBrac{\dsum_{i=1}^{n} Y_i}^2
        }}
    \end{aligned}
    \]

    \item estimator for Pearson’s correlation

    \item $r_P$ is typically \textbf{not unbiased}, $\mathbb{E}(r_P) \neq \rho_P$
    
    \item the expectation of the numerator in $r_P$ is unbiased for the numerator in the definition of Pearson’s correlation $\rho_P$

    \item if $\rho_P = 0$, pairs $(X_i, Y_i)$ are i.i.d. \textbf{bivariate normally distributed}

    \item distribution function of $r_P$ is related to the t-distribution

    \item CDF of $r_P\sqrt{n-2}/\sqrt{1-r_P}$ has the CDF of a t-distribution with $n - 2$ degrees of freedom

    \item the asymptotic distribution of $r_P$ is \textbf{normal} when the pairs $(X_i, Y_i)$ are i.i.d. $F_{XY}$ with \textbf{finite fourth central moments}.
    \[
        \sqrt{n}(r_P - \rho _P ) 
        \sim 
        \mathcal{N} (0, n(1 - \rho ^2 _P )^2/(n - 3))
    \]
    
    \item $100\%(1 - \alpha)$ \textbf{confidence interval} by:
    \[\left(
        \dfrac{1}{2} \log\dParenBrac{\dfrac{1+r_P}{1-r_P}} - \dfrac{z_{1-\alpha/2}}{\sqrt{n-3}},
        \dfrac{1}{2} \log\dParenBrac{\dfrac{1+r_P}{1-r_P}} + \dfrac{z_{1-\alpha/2}}{\sqrt{n-3}}
    \right]\]
    
    \begin{enumerate}
        \item $z_{1-p}$ the $p$th \textbf{upper quantile} of the \textbf{standard normal distribution function}
    		
    \end{enumerate}
    
    \item transforming limits to original scale using the inverse transformation $[\exp{(2x)} - 1]/[\exp{(2x)} + 1]$ of the Fisher $z$-transformation:
    \[\left(
        \dfrac{
            \exp{2[z_{r_P} - z_{1-\alpha/2}/\sqrt{n - 3}]} - 1
        }{
            \exp{2[z_{r_P} - z_{1-\alpha/2}/\sqrt{n - 3}]} + 1
        },
        \dfrac{
            \exp{2[z_{r_P} + z_{1-\alpha/2}/\sqrt{n - 3}]} - 1
        }{
            \exp{2[z_{r_P} + z_{1-\alpha/2}/\sqrt{n - 3}]} + 1
        }
    \right]\]

    \item If the observed data is not normal and the sample size is relatively large, the \textbf{asymptotic confidence interval} may be applied directly on the product-moment estimator:
    \[\left(
        r_P - \dfrac{z_{1-\alpha/2}(1 - r ^2_P )}{\sqrt{n-3}},
        r_P + \dfrac{z_{1-\alpha/2}(1 - r ^2_P )}{\sqrt{n-3}}
    \right]\]


\end{enumerate}


\subsection{Fisher $z$-transformation \cite{ism-1}}\label{Fisher z-transformation}

\begin{enumerate}[itemsep=0.2cm]
    \item For any value of $\rho_P$, but still assuming that the pairs $(X_i, Y_i)$ are \textbf{bivariate normally distributed}: $z_{r_P} = 0.5[\log(1 + r_P ) - log(1 - r_P )]$ is approximately \textbf{normally distributed}

    \begin{enumerate}
        \item Mean: $0.5[\log(1 + \rho_P ) - \log(1 - \rho_P )]$

        \item Variance: $1/(n - 3)$ 

    \end{enumerate}

    \item This transformation is referred to as the Fisher z-transformation.

    \item Fisher z-transformation is also frequently applied when the underlying data is not normally distributed.    

\end{enumerate}


\section{Kendall’s Tau Correlation ($\tau_K$) \cite{ism-1}} \label{Multivariate Distributions: Kendall’s Tau Correlation}

\begin{enumerate}
    \item[] $\begin{aligned}
        \tau_K 
        &= Pr ((X_2 - X_1)(Y_2 - Y_1) > 0) - Pr ((X_2 - X_1)(Y_2 - Y_1) < 0) \\
        &= 2 Pr ((X_2 - X_1)(Y_2 - Y_1) > 0) - 1 \\
        &= 4 Pr(X_1 < X_2, Y_1 < Y_2) - 1.
    \end{aligned}$

    \item $-1 \leq \tau_K \leq 1$
    \begin{enumerate}
        \item minimum $-1$ is attained when $F_{XY}$ is equal to the \textbf{Fréchet lower bound}:
            \[
                F_{XY}(x, y) = \max\dCurlyBrac{F_X(x) + F_Y(y) - 1, 0}
            \]

        \item maximum $1$ is attained when $F_{XY}$ is equal to the \textbf{Fréchet upper bound}:
            \[
                F_{XY}(x, y) = \min\dCurlyBrac{F_X(x), F_Y(y)}
            \]

        \item When $X$ and $Y$ are \textbf{independent}, Kendall’s tau is equal to \textbf{zero}.
        \begin{enumerate}
            \item if $X$ and $Y$ are independent, then all four random variables are independent, since we assumed that the two pairs $(X_1, Y_1)$ and $(X_2, Y_2)$ were already independent
            
            \item This makes $D_X$ and $D_Y$ independent and this implies that the probability that $D_X D_Y > 0$ is equal to $0.5$, which makes Kendall’s tau equal to zero
        \end{enumerate}
    \end{enumerate}

    \item Pearson’s correlation coefficient of $D_X$ and $D_Y$ is given by $CORR(D_X, D_Y) = CORR(X, Y)$

    \item It is a measure of \textbf{concordance}

    \item measures the difference between concordant and discordant pairs

    \item Kendall’s tau measures the strength between the dependency of $D_X$ and $D_Y$

    \item \textit{Maurice Kendall} defined his correlation coefficient on data (thus as an \textbf{estimator} and \textbf{not as a parameter})

    \item population parameter it represents is obvious

    \item fits very well with the continuous \textbf{FGM distributions} and the \textbf{Fréchet family} of distribution function
    
    \item Kendall’s tau estimator is \textbf{computationally more intensive}, since it requires the comparisons of pairs with all other pairs. 
    
    \item on really large data, Kendall’s tau may lead to numerical issues

    \item \textbf{CANNOT} be applied to nominal data

\end{enumerate}

\section{Estimator of Kendall’s tau correlation ($r_K$) \cite{ism-1}} \label{Multivariate Distributions: Estimator of Kendall’s tau correlation}

\begin{enumerate}
    \item[] $\begin{aligned}
        r_K
        &= \dfrac{1}{n(n-1)} \dsum_{i=1}^n \dsum_{j=1}^n
            sgn(X_j - X_i)sgn(Y_j - Y_i) \\
        &= \dfrac{2}{n(n-1)} \dsum_{i=1}^{n-1} \dsum_{j=i+1}^n sgn(X_j - X_i)sgn(Y_j - Y_i)
    \end{aligned}$

    \item estimator depends only on the signs of $X_j - X_i$ and $Y_j - Y_i$

    \item estimator is independent of monotone transformations (increasing or decreasing functions) of the data

    \item If we apply the estimator on $(\psi_1(X_1), \psi_2(Y_1)), (\psi_1(X_2), \psi_2(Y_2)), \cdots , (\psi_1(X_n), \psi_2(Y_n))$, when both $\psi_1$ and $\psi_2$ are increasing or both decreasing functions, we obtain the exact same estimator

    \item distribution function of this estimator is approximately normal when sample size converges to infinity ($\infty$)

    \item variance of $r_K : [2(2n + 5)]/[3n(n - 1)]$  the pairs are uncorrelated ($\tau_K = 0$)

    \item the variance of $r_K$ is bounded from above with $VAR(r_K) \leq 2(1 - \tau_K^2)/n$

    \item Fisher $z$-transformation
    \begin{enumerate}
        \item variance of $r_K$ in the transformed Fisher $z$ scale has been determined at $0.437/(n - 4)$

        \item Confidence interval:
        \[\left(
            z_{r_K} - z_{1-\alpha/2}\sqrt{0.437/(n - 4)},
            z_{r_K} + z_{1-\alpha/2}\sqrt{0.437/(n - 4)}
        \right]\]

    \end{enumerate}

    \item $z_{r_K} = 0.5[log(1 + r_K) - log(1 - r_K)]$ \& $z_p$ the $p$th quantile of the \textbf{standard normal distribution}, original scale:
    \[\left(
        \dfrac{
            exp(z_{r_K} - z_{1-\alpha/2}\sqrt{0.437/(n - 4)}) - 1
        }{
            exp(z_{r_K} - z_{1-\alpha/2}\sqrt{0.437/(n - 4)}) + 1
        },
        \dfrac{
            exp(z_{r_K} + z_{1-\alpha/2}\sqrt{0.437/(n - 4)}) - 1
        }{
            exp(z_{r_K} + z_{1-\alpha/2}\sqrt{0.437/(n - 4)}) + 1
        }
    \right]\]

    
\end{enumerate}


\section{Spearman’s Rho Correlation ($\rho_S$) \cite{ism-1}} \label{Multivariate Distributions: Spearman’s Rho Correlation}

\begin{enumerate}
    \item use three i.i.d. pairs $(X_1, Y_1)$, $(X_2, Y_2)$, and $(X_3, Y_3)$ of random variables having joint distribution function $F_{XY}$

    \item only the estimator for this dependency parameter
    \[
        \rho_S = 2[Pr((X_1 - X_2)(Y_3 - Y_1) > 0) - Pr((X_1 - X_2)(Y_3 - Y_1) < 0)]
    \]

    \item[] $-1 \leq \rho_S \leq 1$
    
    \item Spearman’s rho quantifies concordance of 
    \begin{enumerate}
        \item change in one dimension from the first to the second observation

        \item change in the second dimension from the first to the third observation
    \end{enumerate}
    
    \item when $X$ and $Y$ are independent, Spearman’s rho correlation coefficient becomes equal to zero
    
    \item fits very well with the \textbf{FGM} and \textbf{Fréchet families} of distribution functions
    
    \item \textbf{CANNOT} be applied to \textbf{nominal data}
\end{enumerate}

\section{Spearman’s rank correlation ($r_S$) \cite{ism-1}} \label{Multivariate Distributions: Spearman’s rank correlation}

\begin{enumerate}
    \item estimator for Spearman’s rho

    \item pairs of random variables $(X_1, Y_1), (X_2, Y_2), \cdots , (X_n, Y_n)$, we can define the ranks for the $x$ and $y$ coordinates separately

    \item The rank $R_k^X$ of $X_k$ is the position of $X_k$ in the ordered values $X_{(1)}, X_{(2)}, \cdots, X_{(n)}$ from small to large

    \item we define the rank $R_k^Y$ for $Y_k$ among the random variables $Y_{(1)}, Y_{(2)},\cdots, Y_{(n)}$

    \item if the same value occurs multiple times they all receive the same rank. This is called a \textbf{tie}.

    \item translate the pair $(X_k, Y_k)$ to a pair of ranks $(R_k^X, R_k^Y)$
    \[
        \hfill
        r_S
        =\dfrac{
            \dsum_{i=1}^{n} 
            (R_i^X - \bar{R}^X)(R_i^Y - \bar{R}^Y)
        }{\sqrt{
            \dsum_{i=1}^{n} 
            (R_i^X - \bar{R}^X)^2
            \dsum_{i=1}^{n} 
            (R_i^Y - \bar{R}^Y)^2
        }}
        \hfill
        r_S
        = 1 - 6\dsum_{i=1}^{n}
        \dfrac{(R_i^Y - R_i^X)^2}{n^3 - n}
        \hfill
    \] 

    \item average ranks for variables:
    \[
        \hfill
        \bar{R}^X = \dfrac{1}{n} \dsum_{k=1}^{n} R_k^X
        \hfill
        \bar{R}^Y = \dfrac{1}{n} \dsum_{k=1}^{n} R_k^Y
        \hfill
    \]

    \item total number of ranks in a sample of $n$ observations = $n(n + 1)/2$

    \item Since Spearman’s rank correlation depends on the ranks of the variables $x$ and $y$, the estimator is independent of monotonic transformations (increasing or decreasing functions) of the data.

    \item Spearman rho’s estimator $r_S$ on $(X_1, Y_1), (X_2, Y_2), \cdots , (X_n, Y_n)$ is exactly the same as Spearman rho’s estimator $r_S$ on the transformed pairs $(\psi_1(X_1), \psi_2(Y_1)), (\psi_1(X_2), \psi_2(Y_2)), \cdots , (\psi_1(X_n), \psi_2(Y_n))$, when $\psi_1$ and $\psi_2$ are both increasing or both decreasing functions.

\end{enumerate}

\section{Relation Between Kendall’s tau ($\tau_K$) \& Spearman’s Rho ($\rho_S$) \cite{ism-1}}\label{Relation Between Kendall’s tau and Spearman’s Rho}

\[
    \hfill
    -1 \leq 3\tau_K - 2\rho_S \leq 1
    \hfill
\]


\section{Cohen’s Kappa Statistic ($\kappa_C$) \cite{ism-1}} \label{Multivariate Distributions: Cohen’s Kappa Statistic}

\begin{enumerate}
    \item[] $X, Y$: random variables ($X, Y \in \dCurlyBrac{1, 2, 3,\cdots, K}$) 

    \item[] PDF parameters: $p_{11}, p_{12},\cdots, p_{1K}, p_{21}, p_{22},\cdots, p_{2K},\cdots, p_{K1}, p_{K2},\cdots, p_{KK}$ 
    
    \item uses another measure of \textbf{association}, referred to as a measure of agreement

    \item typically developed for nominal data, it has been applied to ordinal random variables as well, including binary random variables

    \item[] \[
        \hfill
        Pr(X=x,Y=y) = p_{xy} \geq 0
        \hfill
        \dsum_{x=0}^K \dsum_{y=0}^K p_{xy} = 1
        \hfill
    \] 

    \item probability $p_{xy}$ represents the probability that a unit is classified by the first rater in class $x$ and by the second rater in class $y$

    \item probability $p_{kk}$ indicates the probability that both raters classify a unit in the same class $k$

    \item when $p_{xy} = 0$ for every $x \neq y$, there will be no difference in classification between the two raters

    \item when the ratings are independent, $Pr(X = x, Y = y) = f_X(x)f_Y(y)$

    \item $P_O = \dsum_{k=1}^K p_{kk}$ represents the probability that both raters classify units in the same classes
    \begin{enumerate}
        \item When it is equal to $1$, there is perfect agreement.

        \item the expected probability that both raters classify a unit in the same class $k$ is equal to $f_X(k)f_Y(k)$
    \end{enumerate}

    \item $P_E = \dsum_{k=1}^K f_X(k)f_Y(k)$ represents probability of correctly classifying units in the same classes, based on independent ratings

    \item When $X$ and $Y$ are further apart, $\dabs{X - Y} > 1$, this seems to be a more serious \textbf{misclassification} then when $X$ and $Y$ just differ one class, $\dabs{X - Y} = 1$
    \[
        \hfill
        \kappa_C = \dfrac{P_O - P_E}{1 - P_E}
        \hfill
        (0 \leq \kappa_C \leq 1)
    \]

    \item When there is perfect agreement ($P_O = 1$), the kappa statistic reaches its \textbf{maximum} at the value of one $(\kappa_C = 1)$

    \item when the ratings are independent, the kappa statistic is equal to $\kappa_C = 0$

    \item Cohen’s kappa statistic should \textbf{NOT} be used when particular outcomes of $X$ and $Y$ are rare

\end{enumerate}

\begin{customTableWrapper}{1}
\begin{table}[H]
    \centering
    \begin{tabular}{l l l l l l}
    0.80 & < & $\kappa_C$ & $\leq$ & 1.00 &  \textbf{High} agreement\\
    0.60 & < & $\kappa_C$ & $\leq$ & 0.80  & \textbf{Substantial} agreement \\
    0.40 & < & $\kappa_C$ & $\leq$ & 0.60 &  \textbf{Moderate} agreement \\
    0.20 & < & $\kappa_C$ & $\leq$ & 0.40 &  \textbf{Fair} agreement \\
    0   & < & $\kappa_C$ & $\leq$ & 0.20 &  \textbf{Poor} agreement
    \end{tabular}
\end{table}
\end{customTableWrapper}


\section{Estimator of Cohen’s Kappa ($\hat{\kappa}_C$) \cite{ism-1}} \label{Multivariate Distributions: Estimator of Cohen’s Kappa}


\begin{enumerate}
    \item $(X_1, Y_1), (X_2, Y_2), \cdots , (X_n, Y_n)$ of i.i.d. bivariate random variables as copies of $(X, Y)$ that can take their values in $\dCurlyBrac{1, 2,\cdots, K}$
    \[
        \hfill
        N_{xy} = \dsum_{i=1}^{n}
        \mathbbm{1}_{(x)}(X_i) \mathbbm{1}_{(y)}(Y_i)
        \hfill
        \dsum_{x=1}^{K} \dsum_{y=1}^{K} N_{xy} = n
        \hfill
        N_{x\cdot} = \dsum_{y=1}^{K} N_{xy}
        \hfill
        N_{\cdot y} = \dsum_{x=1}^{K} N_{xy}
        \hfill
    \]

    \item distribution function of $N_{xy}$ is binomial with parameters $n$ and 
    \[
        p_{xy} = Pr(X = x, Y = y)
    \]

    \item the multivariate PMF for $N_{11}, N_{12},\cdots, N_{1K}, N_{21}, N_{22},\cdots, N_{2K} , \cdots, N_{K1}, N_{K2},\cdots, N_{KK}$ has a multinomial distribution function given by:
    \[
        Pr(N_{11}=n_{11},\cdots,N_{KK}=n_{KK})
        = n!\dprod_{x=1}^{K} \dprod_{y=1}^{K}
        \dParenBrac{ \dfrac{p_{xy}^{n_{xy}}}{n_{xy}!} }
    \]

    \item estimators: $
        \hfill
        \hat{\kappa}_C = \dfrac{\hat{p}_O - \hat{p}_E}{1 - \hat{p}_E}
        \hfill
        \hat{p}_O = \dfrac{1}{n} \dsum_{k=1}^K N_{kk}
        \hfill
        \hat{p}_O = \dfrac{1}{n^2} \dsum_{k=1}^K 
        N_{k\cdot}N_{\cdot k}
        \hfill
    $
    
    \item $
        VAR(\hat{\kappa}_C)
        \approx \dfrac{p_O(1 - p_O)}{n(1 - p_E)^2}
    $

    \item $100\%(1 - \alpha)$ confidence interval:
    \[\left(
        \hat{\kappa}_C - z_{1-\alpha/2}
        \dfrac{\sqrt{\hat{p}_O(1-\hat{p}_O)}}{(1 - \hat{p}_E)\sqrt{n}},
        \hat{\kappa}_C + z_{1-\alpha/2}
        \dfrac{\sqrt{\hat{p}_O(1-\hat{p}_O)}}{(1 - \hat{p}_E)\sqrt{n}}
    \right]\]
    $z_p$ the $p$th quantile of the \textbf{standard normal distribution function}

    
\end{enumerate}

\section{Pearson’s chi-square statistic (Nominal Statistic) ($\chi_P^2$) \cite{ism-1}} \label{Multivariate Distributions: Pearson’s chi-square statistic (Nominal Statistic)}

\begin{enumerate}
    \item[] $X, Y$: nominal/ ordinal/ binary random variables
    \[
        \hfill
        X \in \dCurlyBrac{1, 2,\cdots, K}
        \hfill
        Y \in \dCurlyBrac{1, 2,\cdots, M}
        \hfill
    \]

    \item[] $(X_1, Y_1), (X_2, Y_2), \cdots , (X_n, Y_n)$ are i.i.d. with CDF $F_{XY}$

    \item represents a measure of departure from independence

    \item $
        N_{xy}
        = \dsum_{i=1}^{n} 
        \mathbbm{1}_{(x)}(X_i)
        \mathbbm{1}_{(y)}(Y_i)
        \hfill
        (x,y) \in [1,\cdots,K] \times [1,\cdots,M]
    $\\
    $N_{xy}$ represent the frequencies of the cells in a $K \times M$ contingency table
    
    \item row and column totals: $
        \hfill
        N_{x\cdot} = \dsum_{y=1}^{K} N_{xy}
        \hfill
        N_{\cdot y} = \dsum_{x=1}^{K} N_{xy}
        \hfill
    $

    \item joint PMF $Pr(X = x, Y = y)$ can be estimated by $N_{xy}/n$

    \item marginal PMFs $Pr(X = x)$ and $Pr(Y = y)$ can be estimated by $N_{x\cdot}/n$ and $N_{\cdot y}/n$, respectively

    \item quantifies the difference between the observed numbers $N_{xy}$ and their expected numbers $N_{x\cdot}N_{\cdot y}/n$ based on independence

    \item $
        \hfill
        \chi_P^2 = \dsum_{x=1}^{K} \dsum_{y=1}^{M}
        \dfrac{[N_{xy} - N_{x\cdot}N_{\cdot y}/n]^2}{N_{x\cdot}N_{\cdot y}/n}
        \hfill
        \chi_P^2 = \dsum_{x=1}^{K} \dsum_{y=1}^{M}
        \dfrac{[nN_{xy} - N_{x\cdot}N_{\cdot y}]^2}{N_{x\cdot}N_{\cdot y}}
        \hfill
    $

    \item \textbf{larger} the value for Pearson’s chi-square statistic, the \textbf{stronger} the association between the two random variables $X$ and $Y$

    \item $(K - 1)(M - 1)$ degrees of freedom when $X$ and $Y$ are independent

    \item cannot be viewed as a correlation coefficient (If we multiply all frequencies in the contingency table with a constant, Pearson’s chi-square increases with the same constant. Thus Pearson’s chi-square statistic is not properly normalized to be viewed as a proper association statistic)

\end{enumerate}


\section{Pearson’s squared phi-coefficient (Nominal Statistic) ($\phi^2$) \cite{ism-1}} \label{Multivariate Distributions: Pearson’s squared phi-coefficient (Nominal Statistic)}

\begin{enumerate}
    \item normalized \textbf{pearson’s chi-squared statistic} (SEE: \fullref{Multivariate Distributions: Pearson’s chi-square statistic (Nominal Statistic)})

    \item sample size $n$ with $2 \times  2$ contingency tables $(K = M = 2)$

    \item $
        \hfill
        \phi^2 = \dfrac{\chi_P^2}{n}
        \hfill
        \chi_P^2 = n \dfrac{
            [N_{11}N_{22} - N_{12}N_{21}]^2
        }{
            N_{1\cdot}N_{2\cdot}N_{\cdot 1}N_{\cdot 2}
        }
        \hfill
    $

    \item $
        \hfill
        \phi = \dfrac{
            \dabs{N_{11}N_{22} - N_{12}N_{21}}
        }{
            \sqrt{N_{1\cdot}N_{2\cdot}N_{\cdot 1}N_{\cdot 2}}
        }
        \hfill
    $

    \item $\phi$ is equal to the absolute value of Pearson’s product-moment estimator applied on the binary pairs $(X_i, Y_i)$

    \item Pearson’s product-moment estimator: $
        \hfill
        \rho_P = \dfrac{
            \dabs{N_{11}N_{22} - N_{12}N_{21}}
        }{
            \sqrt{N_{1\cdot}N_{2\cdot}N_{\cdot 1}N_{\cdot 2}}
        }
        \hfill
    $

    \item The absolute value of $\rho_P$ is now equal to $\phi$
\end{enumerate}


\section{Cramér’s V association measure (Nominal Statistic) ($V$) \cite{ism-1}} \label{Multivariate Distributions: Cramer’s V association measure (Nominal Statistic)}


\begin{enumerate}
    \item based on \textbf{pearson’s chi-squared statistic} (SEE: \fullref{Multivariate Distributions: Pearson’s chi-square statistic (Nominal Statistic)})

    \item general $K \times  M$ contingency tables that would have its values in $[0, 1]$

    \item $
        V = \sqrt{\dfrac{\chi_P^2}{
        n\min\dCurlyBrac{K-1,M-1}
        }}
    $

    \item Cramér’s V is equal to Pearson’s $\phi$ coefficient when either the $X$ or the $Y$ variable has just two levels

    \item Cramér’s $V$ is the more general measure of nominal association, since it is properly normalized for all contingency tables

    \item The maximum attainable value for $V$ can be substantially smaller if we keep the row and column totals fixed to the totals that we have observed

    \item under the marginal constraints we cannot make the association any stronger (in this direction)

    \item the fact that it is not close to one does not necessarily imply that the relationship is very weak because the marginal distribution “limits” the possible values of $\phi$ (or other measures of relationship strength)

    \item for nominal random variables it does provide similar interpretations as the sample correlations and often $\phi$ or $V$ are reported together with $X^2$ and a $p$-value of the null-hypothesis test that there is no dependence between variables
\end{enumerate}


\section{Pearson’s chi-square ($\chi_P^2$), Pearson’s phi-coefficient ($\phi$), and Cramér’s V ($V$) \cite{ism-1}} \label{Multivariate Distributions: Pearson’s chi-square, Pearson’s phi-coefficient, and Cramér’s V}

\begin{enumerate}
    \item meant for contingency tables, they can also be applied to continuous variables $X$ and $Y$

    \item $X$ and $Y$ must be transformed to categorical variables first (forming non-overlapping intervals)

    \item $K$ intervals for the variable $X: (-\infty, \alpha_1], (\alpha_1, \alpha_2],\cdots,(\alpha_{K-2}, \alpha_{K-1}]$, and $(\alpha_{K-1},\infty)$, with $\alpha_1, \alpha_2,\cdots, \alpha_K$ threshold values that are selected by ourselves

    \item $M$ intervals for the variable $Y: (-\infty, \beta_1], (\beta_1, \beta_2],\cdots,(\beta_{M-2}, \beta_{M-1}]$, and $(\beta_{M-1},\infty)$, with $\beta_1, \beta_2,\cdots, \beta_M$ threshold values that are selected by ourselves

    \item The continuous data $(X_1, Y_1), (X_2, Y_2), \cdots , (X_n, Y_n)$ can then be summarized by $N_{xy}$ 

    \item represent the number of pairs $(X_i, Y_i)$ that falls in the set 
    \[
        (\alpha_{x-1}, \alpha_x] \times (\beta_{y-1}, \beta_y] 
        \hfill
        (\alpha_0 = \beta_0 = -\infty \text{ and } \alpha_K = \beta_M = \infty)
    \]

    \item choice of the number of levels and the choice of thresholds can have a strong influence on the calculation of nominal associations
\end{enumerate}


\section{Goodman and Kruskal’s Gamma (Ordinal Statistic) ($\gamma$) \cite{ism-1}} \label{Multivariate Distributions: Goodman and Kruskal’s Gamma (Ordinal Statistic)}


\begin{enumerate}
    \item[] $
        \hfill
        X \in \dCurlyBrac{1, 2,\cdots, K}
        \hfill
        Y \in \dCurlyBrac{1, 2,\cdots, M}
        \hfill
    $

    \vspace{0.2cm}
    \item[] $
        \gamma
        = \dfrac{
            Pr((X_2 - X_1)(Y_2 - Y_1) > 0) - Pr((X_2 - X_1)(Y_2 - Y_1) < 0)
        }{
            Pr((X_2 - X_1)(Y_2 - Y_1) > 0) + Pr((X_2 - X_1)(Y_2 - Y_1) < 0)
        }
    $

    \vspace{0.2cm}
    \item Potentially, Pearson’s rho, Spearman’s rho, and Kendall’s tau estimators may seem to be suitable for this type of data, but they are not ideal

    \begin{enumerate}
        \item Pearson’s rho will treat the values $1, 2, \cdots, K$ for $X$ and the values $1, 2, \cdots, M$ for $Y$ as numerical, while these numbers are somewhat arbitrary

        \item Spearman’s rho and Kendall’s tau compare the ordinal values with each other, but in many comparisons the values cannot be ordered.
    \end{enumerate}

    \item Estimator: $
        \hfill
        G = \dfrac{N_C - N_D}{N_C + N_D}
        \hfill
    $

    \begin{enumerate}
        \item $N_C$ 
        = number of concordant pairs 
        \hfill
        $
            N_C
            =\dsum_{i=1}^n \dsum_{j=1}^n
            \mathbbm{1}_{(0,\infty)} ((X_j - X_i)(Y_j - Y_i))
        $
        \hfill
    
        \item $N_D$ 
        = number of discordant pairs 
        \hfill
        $
            N_D
            =\dsum_{i=1}^n \dsum_{j=1}^n
            \mathbbm{1}_{(-\infty,0)} ((X_j - X_i)(Y_j - Y_i))
        $
        \hfill
    
        \item sample size $n$ is typically larger than the sum of concordant and discordant pairs, i.e., $n > N_C + N_D$, due to the many ties in data of the $K \times M$ contingency table
    \end{enumerate}

    \item if there are no ties at all, Goodman and Kruskal’s gamma reduces to Kendall’s tau

    \item Distribution function of the estimator G has been studied under the assumption that $(X_1, Y_1), (X_2, Y_2), \cdots , (X_n, Y_n)$ are i.i.d. with distribution function $F_{XY}$

    \item $\sqrt{n}(G - \gamma)$ is asymptotically normal
    \begin{enumerate}
        \item Mean: $0$

        \item variance $
            \leq
            \dfrac{2(1 - \gamma^2)}{
                Pr((X_2 - X_1)(Y_2 - Y_1) > 0) + Pr((X_2 - X_1)(Y_2 - Y_1) < 0)
            }
        $

        \item estimator of variance	: $\dfrac{2(1 - G^2)}{N_D + N_C}$
    \end{enumerate}

    \item asymptotic confidence intervals:
    $
        \left(
            G - z_{1-\alpha/2}\sqrt{
                \dfrac{2(1-G^2)}{N_D+N_C}
            },
            G + z_{1-\alpha/2}\sqrt{
                \dfrac{2(1-G^2)}{N_D+N_C}
            }
        \right]
    $

    \item $z_p$ the $p$th quantile of the \textbf{standard normal distribution function}
\end{enumerate}


\section{Similarity Indices (Binary Association Statistics) \cite{ism-1}}\label{Multivariate Distributions: Similarity Indices (Binary Association Statistics)}

\begin{enumerate}
    \item $n$ \textbf{binary attributes} on two objects: $(X_1, Y_1), (X_2, Y_2), \cdots , (X_n, Y_n)$ \\
    pairs may not be independent and/or identically distributed with just one CDF $F_{XY}$

    \item measure or quantify the similarity in a large number of attributes from two objects
    \[
        N_{xy}
        = \dsum_{i=1}^{n} 
        \mathbbm{1}_{(x)}(X_i)
        \mathbbm{1}_{(y)}(Y_i)
        \hfill
        (x,y) \in \dCurlyBrac{0,1} \times \dCurlyBrac{0,1}
    \]

    \item we prefer the values $\dCurlyBrac{0, 1}$ for $X$ and $Y$, instead of $\dCurlyBrac{1, 2}$

    \item Similarity measures quantify the dependency between $X$ and $Y$. The \textbf{higher} the value the stronger the dependency and \textbf{more similar} the two objects are.

    \item dissimilarity measures can be considered a distance between the objects
\end{enumerate}

\subsection{Gower and Legendre Similarity Indices ($S_\theta$/ $T_\theta$) \cite{ism-1}} \label{Gower and Legendre Similarity Indices}

\begin{enumerate}
    \item[] $\theta > 0$ a constant

    \item $
        S_\theta 
        = \dfrac{N_{00} + N_{11}}{N_{00} + N_{11} + \theta[N_{01} + N_{10}]}
        = \dfrac{\theta}{\theta + S_{SS}}
        \in [0,1]
    $

    \item $
        T_\theta
        = \dfrac{N_{11}}{N_{11} + \theta[N_{01} + N_{10}]}
        = \dfrac{\theta}{\theta + T_k}
        \in [0,1]
    $

    \item dissimilarity measures:
    \begin{enumerate}
        \item $1 - S_\theta$

        \item $1 - T_\theta$
    \end{enumerate}

    \item $S_\theta$ focuses on the similarity of both the absence as well as the presence of attributes

    \item $T_\theta$ focuses only on the presence of attributes

    \item $S_{SS}$ and $T_K$ are not properly normalized. They can be larger than $1$ and are essentially \textbf{unbounded}.

\end{enumerate}

\subsection{Sokal \& Sneath(3) ($S_{SS}$) \cite{ism-1}}\label{Sokal and Sneath(3)}\index{Sokal and Sneath(3)}

\begin{enumerate}
    \item[] $S_{SS} = \dfrac{N_{00} + N_{11}}{N_{01} + N_{10}}$

\end{enumerate}

\subsection{Kulcynski ($T_k$) \cite{ism-1}}\label{Kulcynski}

\begin{enumerate}
    \item[] $T_K = \dfrac{N_{11}}{N_{01} + N_{10}}$
\end{enumerate}

\begin{customTableWrapper}{1}
\begin{table}[H]
    \begin{tabular}{|c|c|c|c|}
        \hline
        \customTableHeaderColor
        $\theta$ & 0.5 & 1 & 2\\
        
        \hline
        
        S & 
        \begin{minipage}{5cm}
            \vspace{0.2cm}
            Sokal \& Sneath(2) \cite{ism-1} \indexlabel{Sokal and Sneath(2)}\\[2ex]
            \(
                S_{0.5}
                = \dfrac{2[N_{00} + N_{11}]}{2[N_{00} + N_{11}] + N_{01} + N_{10}}
            \)
            \vspace{0.2cm}
        \end{minipage} &
        \begin{minipage}{4.5cm}
            \vspace{0.2cm}
            Sokal \& Michener \cite{ism-1} \indexlabel{Sokal and Michener}\\[2ex]
            \(
                S_1
                = \dfrac{N_{00} + N_{11}}{N_{00} + N_{11} + N_{01} + N_{10}}
            \)
            \vspace{0.2cm}
        \end{minipage} &
        \begin{minipage}{5cm}
            \vspace{0.2cm}
            Roger \& Tanimoto \cite{ism-1} \indexlabel{Roger and Tanimoto}\\[2ex]
            \(
                S_2 = \dfrac{N_{00} + N_{11}}{N_{00} + N_{11} + 2[N_{01} + N_{10}]}
            \)
            \vspace{0.2cm}
        \end{minipage}
        \\
        \hline
        
        T &
        \begin{minipage}{5cm}
            \vspace{0.2cm}
            Czekanowsk \cite{ism-1} \indexlabel{Czekanowsk}\\[2ex]
            \(
                T_{0.5}
                = \dfrac{2N_{11}}{2N_{11} + N_{01} + N_{10}}
            \)
            \vspace{0.2cm}
        \end{minipage} &
        \begin{minipage}{4.5cm}
            \vspace{0.2cm}
            Jaccard \cite{ism-1} \indexlabel{Jaccard}\\[2ex]
            \(
                T_1
                = \dfrac{N_{11}}{N_{11} + N_{01} + N_{10}}
            \)
            \vspace{0.2cm}
        \end{minipage} &
        \begin{minipage}{5cm}
            \vspace{0.2cm}
            Sokal \& Sneath(1)  \cite{ism-1} \indexlabel{Sokal and Sneath(1)}\\[2ex]
            \(
                T_2 = \dfrac{N_{11}}{N_{11} + 2[N_{01} + N_{10}]}
            \)
            \vspace{0.2cm}
        \end{minipage}
        \\
        \hline
        
    \end{tabular}
\end{table}
\end{customTableWrapper}


\section{L family \cite{ism-1}}\label{L family}

\begin{enumerate}
    \item[] $S = \lambda + \mu(N_{00} + N_{11})$

    \item $\lambda$ and $\mu$ can only be functions of the row and column totals, i.e., functions of $N_{0\cdot}$, $N_{1\cdot}$, $N_{\cdot 0}$, and $N_{\cdot 1}$

    \item Jaccard similarity measure is not contained in this family, but the Czekanowski index is contained in this family as well as the Sokal \& Michener index 

    \item This class also contains Cohen’s kappa statistic

\end{enumerate}

\begin{customTableWrapper}{2.7}
\begin{table}[H]
    \centering
    \begin{tabular}{|l|l|l|}
        \hline
        \customTableHeaderColor
        & $\lambda$ & $\mu$ \\
        \hline
    
        Sokal \& Michener \cite{ism-1} &  $0$ & $\dfrac{1}{n}$ \\
        \hline
    
        Czekanowski \cite{ism-1} & $1 - \dfrac{1}{N_{1\cdot} + N_{\cdot 1}}$ & $\dfrac{1}{N_{1\cdot} + N_{\cdot 1}}$ \\
        \hline

        Kappa \cite{ism-1} & $-\dfrac{N_{1\cdot }N_{\cdot 1} + N_{0\cdot }N_{\cdot 0}}{N_{1\cdot }N_{\cdot 0} + N_{\cdot 1}N_{0\cdot }}$ & $\dfrac{1}{N_{1\cdot }N_{\cdot 0} + N_{\cdot 1}N_{0\cdot }}$ \\
        \hline

        
    \end{tabular}
\end{table}
\end{customTableWrapper}


\subsection{corrected index for similarity due to chance \cite{ism-1}} \label{corrected index for similarity due to chance}

\begin{enumerate}
    \item[] $CS = \dfrac{S - E(S)}{1 - E(S)}$
    \hfill
    (S: similarity measure)

    \item after correction these three indices (Czekanowski index, the Sokal \& Michener index, and Cohen’s kappa statistics) are all equivalent

    \item When a chance corrected measure is more appropriate, the similarity measure $S$ can be used
\end{enumerate}

\section{Yule’s Q statistic (coefficient of association) ($Q$) \cite{ism-1}}\label{Yule’s Q statistic (coefficient of association)}

\begin{enumerate}
    \item[] $Q = \dfrac{N_{00}N_{11} - N_{01}N_{10}}{N_{00}N_{11} + N_{01}N_{10}} \in [-1,1]$

    \item $Q = 0$, there is no association

    \item Yule’s Q is a special case of Goodman \& Kruskal’s $\gamma$ statistic applied to a $2 \times 2$ contingency table: $Q = \dfrac{\hat{OR} - 1}{\hat{OR} + 1}$

    \item Yule’s Q is a monotone function of the odds ratio, which makes Yule’s Q an attractive measure

    \item It transforms the odds ratio to a measure that is in line with correlation coefficients

    \item if we randomly eliminate attributes from one object, say remove half of all the attributes from object 1, then both $N_{11}$ and $N_{10}$ would reduce by a factor $2$, but Yule’s Q would not reduce

    \item Yule’s Q statistic is robust against the number of features that are present in one object, a characteristic that does not hold for the similarity measures

    \item Confidence intervals on Yule’s Q can easily be determined by using the confidence limits of the odds ratio

    \item when it is important that the measure is robust against the number of features that could be present, Yule’s Q may seem an appropriate choice
\end{enumerate}

\section{Distance Measure ($d$) \cite{ism-1}}

\begin{enumerate}
    \item $d(a, b) \geq 0$

    \item $d(a, a) = 0$

    \item $d(a, b) = d(b, a)$

    \item $d(a, b) + d(b, c) \geq d(a, c)$
\end{enumerate}



\section{Conjugate Prior Distribution \cite{ism-1}} \label{conjugate prior distribution}

\begin{enumerate}
    \item Formally, when $\mathcal{F}$ is a class of sampling distribution functions, and $\mathcal{P}$ is a class of \textbf{prior distribution functions}, then the class $\mathcal{P}$ is called \textbf{conjugate} for $\mathcal{F}$ if:
    \[
        \hfill
        p(\theta|x) \in \mathcal{P} 
        \hfill
        \forall\; 
        p(\cdot|\theta) \in \mathcal{F} 
        \text{ and } 
        p(\cdot) \in \mathcal{P}
    \]

    \item If we choose $\mathcal{P}$ as the class of all distribution functions, then $\mathcal{P}$ is \textbf{always conjugate}. However, in practice we are most interested in \textbf{natural conjugate prior families}, which arise by taking $\mathcal{P}$ to be the set of all densities having the same functional form as the sampling distribution (i.e., the likelihood).
\end{enumerate}































