\chapter{Activation Functions}\label{chapter: Activation Functions}

\section{Definition \cite{wiki-Artificial_neuron, wiki-activation-fn}}

The activation function of a node in an artificial neural network is a function that calculates the output of the node based on its individual inputs and their weights. Nontrivial problems can be solved using only a few nodes if the activation function is nonlinear

\vspace{0.2cm}
\begin{enumerate}
    \item  The transfer functions usually have a \textbf{sigmoid} shape, but they may also take the form of other non-linear functions, piecewise linear functions, or step functions.
    \item They are also often monotonically increasing, continuous, differentiable and bounded. 
    \item  Non-monotonic, unbounded and oscillating activation functions with multiple zeros that outperform sigmoidal and ReLU-like activation functions on many tasks have also been recently explored
\end{enumerate}

\vspace{0.2cm}
\noindent From \cite{wiki-Artificial_neuron} \\

\begin{enumerate}
    \item $n$ is number of inputs
    \item $w_i$ is a vector of synaptic weights
    \item $x_i$ is a vector of inputs
\end{enumerate}
\[
    u=\sum _{i=1}^{n}w_{i}x_{i}
\]

$u$ refers in all cases to the weighted sum of all the inputs to the neuron


\section{Linear combination \cite{wiki-Artificial_neuron}}
\[
    y = u + b
\]

\begin{enumerate}
    \item The output unit is simply the weighted sum of its inputs plus a bias term. A number of such linear neurons perform a linear transformation of the input vector. 
    \item This is usually more useful in the first layers of a network. 
    \item A number of analysis tools exist based on linear models, such as harmonic analysis, and they can all be used in neural networks with this linear neuron. 
    \item The bias term allows us to make affine transformations to the data.
\end{enumerate}







\section*{ALSO SEE}
\begin{enumerate}
    \item \fullref{Step function}
    \item \fullref{Sigmoid function: broad section}
    \item \fullref{Arctangent}
    \item \fullref{Hyperbolic tangent (tanh)}
\end{enumerate}


















































