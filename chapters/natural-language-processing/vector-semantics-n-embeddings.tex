\chapter{Vector Semantics and Embeddings \cite{nlp-1}}

\begin{enumerate}
    \item \textit{Words that occur in similar contexts tend to have similar meanings.}\\ This link between similarity in how words are distributed and similarity in what they mean is called the \textbf{distributional hypothesis} \indexlabel{distributional hypothesis}.

    \item \textbf{Word Embeddings} are numeric representations of words in a lower-dimensional space, capturing semantic and syntactic information. \cite{gfg-word-embeddings-in-nlp}
    
\end{enumerate}

\begin{alternateColorTable}
\begin{table}[h!]
    \centering
    \begin{tabular}{|p{3cm}|p{5cm}|p{5cm}|}
        \hline
        \tableHeaderRow
        \textbf{Feature} & \textbf{Static Word Embeddings} & \textbf{Contextual Word Embeddings} \\
        \hline

        \textbf{Examples} & Word2Vec, GloVe, FastText & ELMo, BERT, GPT, RoBERTa \\
        \hline
        
        \textbf{Context Sensitivity} & Context-independent & Context-dependent \\
        \hline
        
        \textbf{Model Complexity} & Simpler, faster to compute & More complex, computationally intensive \\
        \hline
        
        \textbf{Vector Representation} & Single vector per word & Different vectors for the same word in different contexts \\
        \hline
        
        \textbf{Applications} & Suitable for tasks where context is not crucial & Excels in tasks requiring context understanding (e.g., NER, QA, sentiment analysis) \\
        \hline
    \end{tabular}
    \caption{Comparison of Static and Contextual Word Embeddings \cite{chatgpt}}
\end{table}
\end{alternateColorTable}

\section{Synonymy \cite{nlp-1}}\label{Synonymy}
\begin{enumerate}
    \item when one word has a sense whose meaning is identical to a sense of another word, or nearly identical, we say the two senses of those two words are \textbf{synonyms} \indexlabel{synonyms}.

    \item A more formal definition of synonymy (between words rather than senses) is that two words are synonymous if they are substitutable for one another in any sentence without changing the \textbf{truth conditions} of the sentence, the situations in which the sentence would be true.

    \item One of the fundamental tenets of semantics, called the \textbf{principle of contrast} \indexlabel{principle of contrast}, states that a difference in linguistic form is always associated with some difference in meaning.
\end{enumerate}


\section{Word Similarity \cite{nlp-1}}\label{Word Similarity}
\begin{enumerate}
    \item While words don’t have many synonyms, most words do have lots of similar words.
    
    \item The notion of word \textbf{similarity} is very useful in larger semantic tasks.

    \item Knowing how similar two words are can help in computing how similar the meaning of two phrases or sentences are, a very important component of tasks like question answering, paraphrasing, and summarization.

    
\end{enumerate}

\section{Word Relatedness/ Association \cite{nlp-1}}\label{Word Relatedness/ Association}
\begin{enumerate}
    \item The meaning of two words can be related in ways other than similarity.\\
    \textbf{Example}: Coffee \& cup

    \item One common kind of relatedness between words is if they belong to the same \textbf{semantic field} \indexlabel{semantic field}. A semantic field is a set of words which cover a particular semantic domain and bear structured relations with each other.\\
    \textbf{Example}: hospitals (surgeon, scalpel, nurse, anesthetic, hospital)

\end{enumerate}

\section{Semantic Frames and Roles \cite{nlp-1}} \label{Semantic Frames and Roles}
\begin{enumerate}
    \item A semantic frame is a set of words that denote perspectives or participants in a particular type of event.

    \item Frames have semantic roles (like buyer, seller, goods, money), and words in a sentence can take on these roles.

    \item \textbf{Example}: \textbf{Sam bought the book from Ling} could be paraphrased as \textbf{Ling sold the book to Sam}, and that \textbf{Sam} has the role of the \textbf{buyer} in the frame and \textbf{Ling} the \textbf{seller}.
\end{enumerate}

\section{Connotation \cite{nlp-1}}\label{Connotation}
\begin{enumerate}
    \item Connotation means aspects of a word’s meaning that are related to a writer or reader’s emotions, sentiment, opinions, or evaluations.\\
    \textbf{Example}: some words have positive connotations (wonderful) while others have negative connotations (dreary).

    \item Even words whose meanings are similar in other ways can vary in connotation.\\
    \textbf{Example}: innocent (positive connotation) and naive (negative connotation).

    \item Positive or negative evaluation language is called \textbf{sentiment}\indexlabel{sentiment}.\\
    \textbf{Example}: positive evaluation (great, love) and negative evaluation (terrible, hate). 
\end{enumerate}

\begin{alternateColorTable}
\begin{table}[h!]
    \centering
    \begin{tabular}{|l|p{7cm}|}
        \hline
        \textbf{valence} & pleasantness of the stimulus \\
        \hline
        \textbf{arousal} & intensity of emotion provoked by the stimulus \\
        \hline
        \textbf{dominance} & degree of control exerted by the \\
        \hline
    \end{tabular}
\end{table}
\end{alternateColorTable}

\section{Vector Semantics \cite{nlp-1}}\label{Vector Semantics}
\begin{enumerate}
    \item The idea of vector semantics is to represent a word as a point in a multi-dimensional semantic space that is derived from the distributions of word neighbors.

    \item The meaning of a word can be defined by its \textbf{distribution} in language use, meaning its neighboring words or grammatical environments. 

    \item The idea was that two words that occur in very similar distributions (whose neighboring words are similar) have similar meanings.

    \item Vectors for representing words are called \textbf{embeddings}\indexlabel{word embeddings}. The word “embedding” derives from its mathematical sense as a mapping from one space or structure to another.

\end{enumerate}

SEE: \fullref{Term Frequancy-Inverse Document Frequency (TF-IDF)}













































































