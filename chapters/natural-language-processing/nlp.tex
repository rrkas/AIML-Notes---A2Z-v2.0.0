\chapter{Natural Language Processing (NLP)}\label{chap: nlp}

Natural language processing (NLP) is an interdisciplinary subfield of computer science and artificial intelligence. It is primarily concerned with providing computers the ability to process data encoded in natural language and is thus closely related to information retrieval, knowledge representation and computational linguistics, a subfield of linguistics. Typically data is collected in text corpora, using either rule-based, statistical or neural-based approaches of machine learning and deep learning.\cite{wiki-nlp}

Also known as Computational Linguistics (CL)\index{Computational Linguistics}, Human Language Technology (HLT)\index{Human Language Technology}, Natural Language Engineering (NLE)\index{Natural Language Engineering}.

\section{What is NLP?}
\begin{itemize}
    \item Analyze, understand and generate human languages just like humans do.
    \item Applying computational techniques to language domain.
    \item To explain linguistic theories, to use the theories to build systems that can be of social use.
    \item Borrows from Linguistics, Psycholinguistics, Cognitive Science \& Statistics.
    \item Make computers learn our language rather than we learn theirs.
\end{itemize}

\section{Study of Language}
\begin{longtable}{|p{4cm}|p{6cm}|p{6cm}|}
\caption{Study of Language}\\
\hline
\textbf{Discipline} & \textbf{Typical Problems} & \textbf{Tools} \\
\hline
\endfirsthead

\hline
\textbf{Discipline} & \textbf{Typical Problems} & \textbf{Tools} \\
\hline
\endhead

\hline
\endfoot

\hline
\endlastfoot

Linguists & How do words form phrases and sentences? What constrains the possible meanings for a sentence? & Intuitions about well-formedness and meaning; mathematical models of structure and meaning \\
\hline
Psycholinguists & How do people identify the structure of sentences? How are word and text meanings identified? & Experimental techniques based on measuring human performance; statistical analysis of observations \\
\hline
Philosophers & What is meaning, and how do words and sentences acquire it? How do words identify objects in the world? & Natural language argumentation using intuition about counter-examples; mathematical models (for example, logic and model theory) \\
\hline
Computational Linguists & How is the structure of sentences identified? How can knowledge and reasoning be modeled? How can language be used to accomplish specific tasks? & Algorithms, data structures; formal models of representation and reasoning; AI techniques (search and representation methods) \\
\hline

\end{longtable}

\section{NLP Applications}

\textbf{Examples:}
\begin{itemize}
    \item Question answering
    \item Text Categorization/Routing
    \item Text Mining
    \item Machine (Assisted) Translation
    \item Language Teaching/Learning
    \item Spelling correction
\end{itemize}

\vspace{0.3cm}

\textbf{Areas:}
\begin{itemize}
    \item Text-to-Speech \& Speech recognition
    \item Healthcare
    \item Natural Language Dialogue Interfaces to Databases
    \item Information Retrieval
    \item Information Extraction (\href{http://nlp.stanford.edu:8080/ner/process}{http://nlp.stanford.edu:8080/ner/process}) 
    \item Document Classification
    \item Document Image Analysis
    \item Automatic Summarization (\href{https://quillbot.com/summarize}{https://quillbot.com/summarize}) 
    \item Text Proof-reading – Spelling \& Grammar
    \item Machine Translation
    \item Fake News and Cyberbullying Detection
    \item Monitoring Social Media Using NLP
    \item Plagiarism detection
    \item Look-ahead typing / Word prediction
    \item Question Answering System (\href{http://start.csail.mit.edu/index.php}{http://start.csail.mit.edu/index.php}) 
    \item Sentiment Analysis (\href{https://komprehend.io/sentiment-analysis}{https://komprehend.io/sentiment-analysis}) 
\end{itemize}






















































