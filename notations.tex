\chapter*{Mathematical Notations}

\section*{Brackets}

\begin{customTableWrapper}{1.5}
\begin{longtable}{|p{3cm}|p{12cm}|}
    \hline
    \customTableHeaderColor
    \textbf{Notation} & \textbf{Description/ Usage}\\ \hline
    \endfirsthead

    \hline
    \customTableHeaderColor
    \textbf{Notation} & \textbf{Description/ Usage}\\ \hline
    \endhead

    \hline
    \endfoot

    \hline
    \endlastfoot

    $<\cdots>$/ $\langle \cdots \rangle$ & \tableenumerate{
        \item Inner product
    }\\
    \hline

    $\dCurlyBrac{\cdots}$ & \tableenumerate{
        \item unordered set
        \item unordered basis: $\mathbf{B = \dCurlyBrac{b_1, \cdots , b_n}}$
    }\\
    \hline

    $(\cdots)$ & \tableenumerate{
        \item ordered set
        
        \item ordered basis: $\mathit{B} = \mathbf{(b_1, \cdots , b_n)}$ \fullref{ordered basis}
        
        \item $(a,b)$: range with \textbf{neither limits} included
    }\\
    \hline

    $(\cdots]$ & \tableenumerate{
        \item $(a,b]$: range with only \textbf{upper limit} included
    }\\
    \hline

    $[\cdots)$ & \tableenumerate{
        \item $[a,b)$: range with only \textbf{lower limit} included
    }\\
    \hline

    $[\cdots]$ & \tableenumerate{
        \item $[a,b]$: range with \textbf{both limits} included
        
        \item vectors (\fullref{vectors}) \& matrix
        \[
            \begin{bmatrix}
                x_{11} & \cdots & x_{1n}\\
                \vdots & \ddots & \vdots \\
                x_{m1} & \cdots & x_{mn}
            \end{bmatrix}
        \]

        \item $\mathbf{B = [b_1, \cdots , b_n]}$ is a matrix whose columns are the vectors $\mathbf{b_1, \cdots , b_n}$.
    }\\
    \hline

    $|\cdots|$ & \tableenumerate{
        \item \fullref{abs_value}
    }\\
    \hline

\end{longtable}
\end{customTableWrapper}


\section*{Operator}

\begin{customTableWrapper}{1.5}
\begin{longtable}{|p{3cm}|p{12cm}|}
    \hline
    \customTableHeaderColor
    \textbf{Operator} & \textbf{Description/ Usage}\\ \hline
    \endfirsthead

    \hline
    \customTableHeaderColor
    \textbf{Operator} & \textbf{Description/ Usage}\\ \hline
    \endhead

    \hline
    \endfoot

    \hline
    \endlastfoot

    $+$ & \tableenumerate{
        \item Addition
    }\\
    \hline

    $-$ & \tableenumerate{
        \item Subtraction
        
        \item \textbf{(Set) Difference}: $\mathbb{A}-\mathbb{B}$: It includes all the elements that are in set $\mathbb{A}$ but not in set $\mathbb{B}$.
        $\mathbb{A}-\mathbb{B}=\dCurlyBrac{ x | x \in \mathbb{A} \text{ and } x \not\in \mathbb{B} }$
    }\\
    \hline

    $a/b$ or $\dfrac{a}{b}$ or $\displaystyle\dfrac{a}{b}$ & \tableenumerate{
        \item Division (a divided by b)
        \item Or (a or b)
    }\\
    \hline

    $\times$ / $*$ / $\cdot$ & \tableenumerate{
        \item Multiplication
    }\\
    \hline

    $\cup$ & \tableenumerate{
        \item \textbf{Union}: $\mathbb{A}\cup\mathbb{B}$ means elements that are in either $\mathbb{A}$ or $\mathbb{B}$. 
        $\mathbb{A}\cup\mathbb{B}=\dCurlyBrac{ x | x \in \mathbb{A} \text{ or } x \in \mathbb{B} }$
    }\\
    \hline

    $\cap$ & \tableenumerate{
        \item \textbf{Intersection}: $\mathbb{A}\cup\mathbb{B}$ means elements that are in both $\mathbb{A}$ and $\mathbb{B}$.  
        $\mathbb{A}\cap\mathbb{B}=\dCurlyBrac{ x | x \in \mathbb{A} \text{ and } x \in \mathbb{B} }$
    }\\
    \hline

    $\backslash$ & \tableenumerate{
         \item \textbf{(Set) Difference}: $\mathbb{A}-\mathbb{B}$: It includes all the elements that are in set $\mathbb{A}$ but not in set $\mathbb{B}$.
        $\mathbb{A}\backslash\mathbb{B}=\dCurlyBrac{ x | x \in \mathbb{A} \text{ and } x \not\in \mathbb{B} }$
    }\\
    \hline

    $\cdots$ / $\vdots$ / $\ddots$ & \tableenumerate{
        \item to show a lot of elements
    }\\
    \hline

    $\mapsto$ & \tableenumerate{
        \item "Maps To"
    }\\
    \hline

    $\doubleuptack$ & \tableenumerate{
        \item conditional independence
    }


\end{longtable}
\end{customTableWrapper}


\section*{Symbols}

\begin{customTableWrapper}{1.5}
\begin{longtable}{|p{1.5cm}|p{3cm}|p{10cm}|}
    \hline
    \customTableHeaderColor
    \textbf{Symbol} & \textbf{Name} & \textbf{Description/ Usage}\\ \hline
    \endfirsthead

    \hline
    \customTableHeaderColor
    \textbf{Symbol} & \textbf{Name} & \textbf{Description/ Usage}\\ \hline
    \endhead

    \hline
    \endfoot

    \hline
    \endlastfoot

    
    $\pi$ / $\Pi$ & pi & \tableenumerate{
        \item $\displaystyle\pi \approx \dfrac{22}{7} \text{ or } \dfrac{355}{113} \text{ or } 3.1415926535$

        \item \( \dprod_{i=1}^{n} x_i = x_1 \cdot x_2 \cdots x_n \)
    
        \item \fullref{DRL: Policy}
    }\\
    \hline

    $e$ & Euler's number & \tableenumerate{
        \item \( \displaystyle e = \sum \limits _{n=0}^{\infty }{\frac {1}{n!}}\approx 2.71828 \)

        \item 
        
    }\\
    \hline

    $\alpha$ & Alpha & \tableenumerate{
        \item \fullref{Coordinate vector}
    }\\
    \hline


    $\beta$ & Beta & \tableenumerate{
        \item  
    }\\
    \hline

    $\gamma$ / $\Gamma$ & Gamma & \tableenumerate{
        \item \fullref{Gamma Function}
    }\\
    \hline

    $\sigma$ / $\Sigma$ & Sigma & \tableenumerate{
        \item $\dsum_{i=1}^{n} x_i= x_1 + x_2 + \cdots + c_n$
        \item \fullref{Logistic function}
    }\\
    \hline

    $\phi$ / $\Phi$ & Phi & \tableenumerate{
        \item \fullref{Linear Mappings/ vector space homomorphism/ linear transformation}
    }\\
    \hline

    $\psi$ / $\Psi$ & Psi & \\
    \hline

    $\epsilon$ & epsilon & \tableenumerate{
        \item Exploration: \fullref{Exploration vs. Exploitation}
    }\\
    \hline

    $\eta$ & Eta & \tableenumerate{
        \item Learning Rate
    } \\
    \hline

    $\delta$ / $\Delta$ & Delta & \tableenumerate{
        \item \fullref{Difference Quotient}
        \item \textbf{Symmetric (Set) Difference/ disjunctive union/ set sum}\indexlabel{Symmetric (Set) Difference/ disjunctive union/ set sum}: $\mathbb{A}\Delta\mathbb{B} = (\mathbb{A}-\mathbb{B})\cup(\mathbb{B}-\mathbb{A})$ 
    } \\
    \hline

    $\theta$ / $\Theta$ & Theta & \tableenumerate{
        \item angles:
        \begin{enumerate}
            \item\fullref{Trigonometric functions}
            \item\fullref{Inverse trigonometric functions}
            \item\fullref{Hyperbolic functions}
        \end{enumerate}
    }\\
    \hline

    $\xi$ / $\Xi$ & Xi & \\
    \hline

    $\chi$ & Chi & \\
    \hline

    $\omega$ / $\Omega$ & Omega & \\
    \hline


    $\lambda$ / $\Lambda$ & Lambda & \\
    \hline

    $\nabla$ & Nabla & \tableenumerate{
        \item $\nabla F(x)$: Gradient of the function $F(x)$ wrt $x$
    }\\
    \hline

    $\exists$ & Exists & \tableenumerate{
        \item Example: $\exists a, a<10$ : there exists a such that "a" is less than 10
    }\\
    \hline

    $\forall$ & For all & \tableenumerate{
        \item Example: $\forall a \in \mathbb{A}$ : for all "a" in $\mathbb{A}$ 
    }\\
    \hline

    $\kappa$ & Kappa & \tableenumerate{
        \item Cohen’s Kappa Statistic
    }\\
    \hline

    $\nu$ & Nu & 
    \\
    \hline

\end{longtable}
\end{customTableWrapper}


\section*{Alphabetical Notations}

\begin{customTableWrapper}{1.5}
\begin{longtable}{|p{3cm}|p{12cm}|}
    \hline
    \customTableHeaderColor
    \textbf{Notation} & \textbf{Description/ Usage}\\ \hline
    \endfirsthead

    \hline
    \customTableHeaderColor
    \textbf{Notation} & \textbf{Description/ Usage}\\ \hline
    \endhead

    \hline
    \endfoot

    \hline
    \endlastfoot


    $a$, $b$, $a_i$ & \tableenumerate{
        \item scalar
    }\\
    \hline

    $\mathbf{a}$, $\mathbf{b}$, $\mathbf{a}_i$, $\mathbf{v}^\top$ & \tableenumerate{
        \item \fullref{vectors}
    }\\
    \hline

    $\mathbf{A}$, $\mathbf{B}$, $\mathbf{M}^\top$ & \tableenumerate{
        \item matrix
        \item vector space
    }\\
    \hline

    $\mathbb{A}$, $\mathbb{B}$, $\mathbb{R}$, $\mathbb{R}^{m\times n}$ & \tableenumerate{ 
        \item set

        \item $\mathbb{E}$: Expected value
        
        \item $\mathbb{U}$: Universal set (set theory)
        
        \item $\mathbbm{1}$: \fullref{Indicator function}
    }\\
    \hline



    
\end{longtable}
\end{customTableWrapper}


